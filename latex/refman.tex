\documentclass[twoside]{article}

% Packages required by doxygen
\usepackage{fixltx2e}
\usepackage{calc}
\usepackage{doxygen}
\usepackage[export]{adjustbox} % also loads graphicx
\usepackage{graphicx}
\usepackage[utf8]{inputenc}
\usepackage{makeidx}
\usepackage{multicol}
\usepackage{multirow}
\PassOptionsToPackage{warn}{textcomp}
\usepackage{textcomp}
\usepackage[nointegrals]{wasysym}
\usepackage[table]{xcolor}

% NLS support packages
\usepackage{polski}
\usepackage[T1]{fontenc}

% Font selection
\usepackage[T1]{fontenc}
\usepackage[scaled=.90]{helvet}
\usepackage{courier}
\usepackage{amssymb}
\usepackage{sectsty}
\renewcommand{\familydefault}{\sfdefault}
\allsectionsfont{%
  \fontseries{bc}\selectfont%
  \color{darkgray}%
}
\renewcommand{\DoxyLabelFont}{%
  \fontseries{bc}\selectfont%
  \color{darkgray}%
}
\newcommand{\+}{\discretionary{\mbox{\scriptsize$\hookleftarrow$}}{}{}}

% Page & text layout
\usepackage{geometry}
\geometry{%
  a4paper,%
  top=2.5cm,%
  bottom=2.5cm,%
  left=2.5cm,%
  right=2.5cm%
}
\tolerance=750
\hfuzz=15pt
\hbadness=750
\setlength{\emergencystretch}{15pt}
\setlength{\parindent}{0cm}
\setlength{\parskip}{0.2cm}
\makeatletter
\renewcommand{\paragraph}{%
  \@startsection{paragraph}{4}{0ex}{-1.0ex}{1.0ex}{%
    \normalfont\normalsize\bfseries\SS@parafont%
  }%
}
\renewcommand{\subparagraph}{%
  \@startsection{subparagraph}{5}{0ex}{-1.0ex}{1.0ex}{%
    \normalfont\normalsize\bfseries\SS@subparafont%
  }%
}
\makeatother

% Headers & footers
\usepackage{fancyhdr}
\pagestyle{fancyplain}
\fancyhead[LE]{\fancyplain{}{\bfseries\thepage}}
\fancyhead[CE]{\fancyplain{}{}}
\fancyhead[RE]{\fancyplain{}{\bfseries\leftmark}}
\fancyhead[LO]{\fancyplain{}{\bfseries\rightmark}}
\fancyhead[CO]{\fancyplain{}{}}
\fancyhead[RO]{\fancyplain{}{\bfseries\thepage}}
\fancyfoot[LE]{\fancyplain{}{}}
\fancyfoot[CE]{\fancyplain{}{}}
\fancyfoot[RE]{\fancyplain{}{\bfseries\scriptsize Wygenerowano Śr, 25 mar 2015 13\+:12\+:53 dla Program framework benchmarkujacy dla struktury danych Stos programem Doxygen }}
\fancyfoot[LO]{\fancyplain{}{\bfseries\scriptsize Wygenerowano Śr, 25 mar 2015 13\+:12\+:53 dla Program framework benchmarkujacy dla struktury danych Stos programem Doxygen }}
\fancyfoot[CO]{\fancyplain{}{}}
\fancyfoot[RO]{\fancyplain{}{}}
\renewcommand{\footrulewidth}{0.4pt}
\renewcommand{\sectionmark}[1]{%
  \markright{\thesection\ #1}%
}

% Indices & bibliography
\usepackage{natbib}
\usepackage[titles]{tocloft}
\setcounter{tocdepth}{3}
\setcounter{secnumdepth}{5}
\makeindex

% Hyperlinks (required, but should be loaded last)
\usepackage{ifpdf}
\ifpdf
  \usepackage[pdftex,pagebackref=true]{hyperref}
\else
  \usepackage[ps2pdf,pagebackref=true]{hyperref}
\fi
\hypersetup{%
  colorlinks=true,%
  linkcolor=blue,%
  citecolor=blue,%
  unicode%
}

% Custom commands
\newcommand{\clearemptydoublepage}{%
  \newpage{\pagestyle{empty}\cleardoublepage}%
}


%===== C O N T E N T S =====

\begin{document}

% Titlepage & ToC
\hypersetup{pageanchor=false,
             bookmarks=true,
             bookmarksnumbered=true,
             pdfencoding=unicode
            }
\pagenumbering{roman}
\begin{titlepage}
\vspace*{7cm}
\begin{center}%
{\Large Program framework benchmarkujacy dla struktury danych Stos \\[1ex]\large 1.\+1 }\\
\vspace*{1cm}
{\large Wygenerowano przez Doxygen 1.8.9.1}\\
\vspace*{0.5cm}
{\small Śr, 25 mar 2015 13:12:53}\\
\end{center}
\end{titlepage}
\tableofcontents
\pagenumbering{arabic}
\hypersetup{pageanchor=true}

%--- Begin generated contents ---
\section{Program framework benchmarkujacy dla struktury danych Stos}
\label{index}\hypertarget{index}{}\begin{DoxyAuthor}{Author}
Wojcich Makuch 
\end{DoxyAuthor}
\begin{DoxyDate}{Date}
12.\+05.\+2015 
\end{DoxyDate}
\begin{DoxyVersion}{Version}
2.\+0 program wzbogacono o menu, pozwalajace spradzic poprawnosc algorytmow. wszystkie klasy, metody, zmienne itp. zaimplementowano w jezyku angielskim. 
\end{DoxyVersion}

\section{Indeks klas}
\subsection{Lista klas}
Tutaj znajdują się klasy, struktury, unie i interfejsy wraz z ich krótkimi opisami\+:\begin{DoxyCompactList}
\item\contentsline{section}{\hyperlink{class_c_edge}{C\+Edge} \\*Definicja klasy \hyperlink{class_c_edge}{C\+Edge} definuje krawedz grafu nie zawiera wag }{\pageref{class_c_edge}}{}
\item\contentsline{section}{\hyperlink{class_c_graph}{C\+Graph} \\*Definicja klasy \hyperlink{class_c_graph}{C\+Graph} definije graf skierowany bez wagowaych krawedzi dziedzicy po klasie C\+Benchmark }{\pageref{class_c_graph}}{}
\item\contentsline{section}{\hyperlink{class_c_node}{C\+Node} \\*Definicja klasy \hyperlink{class_c_node}{C\+Node} definiuje pojedynczy wezel grafu }{\pageref{class_c_node}}{}
\item\contentsline{section}{\hyperlink{class_lista}{Lista$<$ typ $>$} \\*Definicja klasy \hyperlink{class_lista}{Lista} Przechowuje obiekt oraz wskaznik na nastepny i pole rozmiar. Zbudowana na szablonie }{\pageref{class_lista}}{}
\item\contentsline{section}{\hyperlink{classqueue}{queue} \\*Definicja klasy queue definicja kolejki A\+D\+T, kolejka typu F\+I\+F\+O zimplementowaa na liscie }{\pageref{classqueue}}{}
\item\contentsline{section}{\hyperlink{classqueue__node}{queue\+\_\+node} \\*Definicja klasy \hyperlink{classqueue__node}{queue\+\_\+node} definicja wezla dla kolejki definiuje pojedynczy element bedacy w kolejkce }{\pageref{classqueue__node}}{}
\end{DoxyCompactList}

\section{Indeks plików}
\subsection{Lista plików}
Tutaj znajduje się lista wszystkich plików z ich krótkimi opisami\+:\begin{DoxyCompactList}
\item\contentsline{section}{\hyperlink{benchmark_8cpp}{benchmark.\+cpp} }{\pageref{benchmark_8cpp}}{}
\item\contentsline{section}{\hyperlink{benchmark_8hh}{benchmark.\+hh} }{\pageref{benchmark_8hh}}{}
\item\contentsline{section}{\hyperlink{main_8cpp}{main.\+cpp} }{\pageref{main_8cpp}}{}
\item\contentsline{section}{\hyperlink{node_8cpp}{node.\+cpp} }{\pageref{node_8cpp}}{}
\item\contentsline{section}{\hyperlink{node_8hh}{node.\+hh} }{\pageref{node_8hh}}{}
\item\contentsline{section}{\hyperlink{_red__black__node_8cpp}{Red\+\_\+black\+\_\+node.\+cpp} }{\pageref{_red__black__node_8cpp}}{}
\item\contentsline{section}{\hyperlink{_red__black__node_8hh}{Red\+\_\+black\+\_\+node.\+hh} }{\pageref{_red__black__node_8hh}}{}
\item\contentsline{section}{\hyperlink{_red__black__tree_8cpp}{Red\+\_\+black\+\_\+tree.\+cpp} }{\pageref{_red__black__tree_8cpp}}{}
\item\contentsline{section}{\hyperlink{_red__black__tree_8hh}{Red\+\_\+black\+\_\+tree.\+hh} }{\pageref{_red__black__tree_8hh}}{}
\item\contentsline{section}{\hyperlink{tree_8cpp}{tree.\+cpp} }{\pageref{tree_8cpp}}{}
\item\contentsline{section}{\hyperlink{tree_8hh}{tree.\+hh} }{\pageref{tree_8hh}}{}
\end{DoxyCompactList}

\section{Dokumentacja klas}
\hypertarget{class_kolejka}{}\subsection{Dokumentacja szablonu klasy Kolejka$<$ typ $>$}
\label{class_kolejka}\index{Kolejka$<$ typ $>$@{Kolejka$<$ typ $>$}}


definicja klasy \hyperlink{class_kolejka}{Kolejka} Zbudowana na tablicy posiada indeksy pokazujace na poczatek i na koniec kolejki. Zbudowana na szablonie.  




{\ttfamily \#include $<$Kolejka.\+hh$>$}

\subsubsection*{Metody publiczne}
\begin{DoxyCompactItemize}
\item 
\hyperlink{class_kolejka_a9e3a347692b91c7c805dbc4c600d3611}{Kolejka} (int ilosc)
\begin{DoxyCompactList}\small\item\em definicja konstruktora z jednym parametrem \end{DoxyCompactList}\item 
\hyperlink{class_kolejka_ae64d506b36a27fdf2c66e53aaa7ae79d}{Kolejka} ()
\begin{DoxyCompactList}\small\item\em definicja konstruktora bezparametrycznego Zeruje rozmiar, ustawia wskaznik tablicy na N\+U\+L\+L, zeruje indeksy. \end{DoxyCompactList}\item 
\hyperlink{class_kolejka_a19d3261c05d90a58feed8340ea995258}{$\sim$\+Kolejka} ()
\begin{DoxyCompactList}\small\item\em definicja destruktora Zwalnia zaalokowana pamiec tablicy. Zeruje indeksy i rozmiar. \end{DoxyCompactList}\item 
int \hyperlink{class_kolejka_a487b41717feb173ed0cac156d143a7f9}{size} () const 
\begin{DoxyCompactList}\small\item\em definicja metody size \end{DoxyCompactList}\item 
void \hyperlink{class_kolejka_a8cf2fbf3b641c1914d63580ce59ec3ef}{enqueue} (typ element)
\begin{DoxyCompactList}\small\item\em definicja metody enqueue \end{DoxyCompactList}\item 
typ \hyperlink{class_kolejka_a7473ad1afce6acba353bfdd78f7c72e0}{dequeue} ()
\begin{DoxyCompactList}\small\item\em definicja metody dequeue \end{DoxyCompactList}\end{DoxyCompactItemize}
\subsubsection*{Atrybuty prywatne}
\begin{DoxyCompactItemize}
\item 
int \hyperlink{class_kolejka_aa533196e5ca1ca0033142a1588000edd}{f}
\item 
int \hyperlink{class_kolejka_aa26f38fd232737021fb82f5d8e86c724}{r}
\item 
typ $\ast$ \hyperlink{class_kolejka_a333beaaaf8ccecbfd4b84a7840d3bd33}{tab}
\item 
int \hyperlink{class_kolejka_a031fd520e5f22daac808dbeeb705832a}{rozmiar}
\end{DoxyCompactItemize}


\subsubsection{Opis szczegółowy}
\subsubsection*{template$<$typename typ$>$class Kolejka$<$ typ $>$}



Definicja w linii 17 pliku Kolejka.\+hh.



\subsubsection{Dokumentacja konstruktora i destruktora}
\hypertarget{class_kolejka_a9e3a347692b91c7c805dbc4c600d3611}{}\index{Kolejka@{Kolejka}!Kolejka@{Kolejka}}
\index{Kolejka@{Kolejka}!Kolejka@{Kolejka}}
\paragraph[{Kolejka}]{\setlength{\rightskip}{0pt plus 5cm}template$<$typename typ $>$ {\bf Kolejka}$<$ typ $>$\+::{\bf Kolejka} (
\begin{DoxyParamCaption}
\item[{int}]{ilosc}
\end{DoxyParamCaption}
)}\label{class_kolejka_a9e3a347692b91c7c805dbc4c600d3611}
max rozmiar tablicy


\begin{DoxyParams}{Parametry}
{\em \mbox{[}ilosc\mbox{]}} & rozmiar alokowanej tablicy Alokuje tablice o zadanym rozmiarze. Ustawia indeksy na 0. \\
\hline
\end{DoxyParams}


Definicja w linii 38 pliku Kolejka.\+hh.

\hypertarget{class_kolejka_ae64d506b36a27fdf2c66e53aaa7ae79d}{}\index{Kolejka@{Kolejka}!Kolejka@{Kolejka}}
\index{Kolejka@{Kolejka}!Kolejka@{Kolejka}}
\paragraph[{Kolejka}]{\setlength{\rightskip}{0pt plus 5cm}template$<$typename typ $>$ {\bf Kolejka}$<$ typ $>$\+::{\bf Kolejka} (
\begin{DoxyParamCaption}
{}
\end{DoxyParamCaption}
)}\label{class_kolejka_ae64d506b36a27fdf2c66e53aaa7ae79d}


Definicja w linii 51 pliku Kolejka.\+hh.

\hypertarget{class_kolejka_a19d3261c05d90a58feed8340ea995258}{}\index{Kolejka@{Kolejka}!````~Kolejka@{$\sim$\+Kolejka}}
\index{````~Kolejka@{$\sim$\+Kolejka}!Kolejka@{Kolejka}}
\paragraph[{$\sim$\+Kolejka}]{\setlength{\rightskip}{0pt plus 5cm}template$<$typename typ $>$ {\bf Kolejka}$<$ typ $>$\+::$\sim${\bf Kolejka} (
\begin{DoxyParamCaption}
{}
\end{DoxyParamCaption}
)}\label{class_kolejka_a19d3261c05d90a58feed8340ea995258}


Definicja w linii 64 pliku Kolejka.\+hh.



\subsubsection{Dokumentacja funkcji składowych}
\hypertarget{class_kolejka_a7473ad1afce6acba353bfdd78f7c72e0}{}\index{Kolejka@{Kolejka}!dequeue@{dequeue}}
\index{dequeue@{dequeue}!Kolejka@{Kolejka}}
\paragraph[{dequeue}]{\setlength{\rightskip}{0pt plus 5cm}template$<$typename typ $>$ typ {\bf Kolejka}$<$ typ $>$\+::dequeue (
\begin{DoxyParamCaption}
{}
\end{DoxyParamCaption}
)}\label{class_kolejka_a7473ad1afce6acba353bfdd78f7c72e0}
\begin{DoxyReturn}{Zwraca}
element z poczatku kolejki 

0 i wyswietla komunikat gdy kolejka jest pusta. zmienia polozenie indeksu poczatku. 
\end{DoxyReturn}


Definicja w linii 126 pliku Kolejka.\+hh.

\hypertarget{class_kolejka_a8cf2fbf3b641c1914d63580ce59ec3ef}{}\index{Kolejka@{Kolejka}!enqueue@{enqueue}}
\index{enqueue@{enqueue}!Kolejka@{Kolejka}}
\paragraph[{enqueue}]{\setlength{\rightskip}{0pt plus 5cm}template$<$typename typ $>$ void {\bf Kolejka}$<$ typ $>$\+::enqueue (
\begin{DoxyParamCaption}
\item[{typ}]{element}
\end{DoxyParamCaption}
)}\label{class_kolejka_a8cf2fbf3b641c1914d63580ce59ec3ef}

\begin{DoxyParams}{Parametry}
{\em \mbox{[}element\mbox{]}} & dodawany element Dodaje element na koniec kolejki. Gdy kolejka jest pelna, powieksza tablice o 5 i przekopiowywuje elementy. zmienia polozenie indeksu konca. \\
\hline
\end{DoxyParams}


Definicja w linii 92 pliku Kolejka.\+hh.

\hypertarget{class_kolejka_a487b41717feb173ed0cac156d143a7f9}{}\index{Kolejka@{Kolejka}!size@{size}}
\index{size@{size}!Kolejka@{Kolejka}}
\paragraph[{size}]{\setlength{\rightskip}{0pt plus 5cm}template$<$typename typ $>$ int {\bf Kolejka}$<$ typ $>$\+::size (
\begin{DoxyParamCaption}
{}
\end{DoxyParamCaption}
) const}\label{class_kolejka_a487b41717feb173ed0cac156d143a7f9}
\begin{DoxyReturn}{Zwraca}
rozmiar ilosci danych przechowywanych w tablicy. 

0, gdy kolejka jest pusta. 
\end{DoxyReturn}


Definicja w linii 76 pliku Kolejka.\+hh.



\subsubsection{Dokumentacja atrybutów składowych}
\hypertarget{class_kolejka_aa533196e5ca1ca0033142a1588000edd}{}\index{Kolejka@{Kolejka}!f@{f}}
\index{f@{f}!Kolejka@{Kolejka}}
\paragraph[{f}]{\setlength{\rightskip}{0pt plus 5cm}template$<$typename typ $>$ int {\bf Kolejka}$<$ typ $>$\+::f\hspace{0.3cm}{\ttfamily [private]}}\label{class_kolejka_aa533196e5ca1ca0033142a1588000edd}


Definicja w linii 18 pliku Kolejka.\+hh.

\hypertarget{class_kolejka_aa26f38fd232737021fb82f5d8e86c724}{}\index{Kolejka@{Kolejka}!r@{r}}
\index{r@{r}!Kolejka@{Kolejka}}
\paragraph[{r}]{\setlength{\rightskip}{0pt plus 5cm}template$<$typename typ $>$ int {\bf Kolejka}$<$ typ $>$\+::r\hspace{0.3cm}{\ttfamily [private]}}\label{class_kolejka_aa26f38fd232737021fb82f5d8e86c724}
poczatek 

Definicja w linii 19 pliku Kolejka.\+hh.

\hypertarget{class_kolejka_a031fd520e5f22daac808dbeeb705832a}{}\index{Kolejka@{Kolejka}!rozmiar@{rozmiar}}
\index{rozmiar@{rozmiar}!Kolejka@{Kolejka}}
\paragraph[{rozmiar}]{\setlength{\rightskip}{0pt plus 5cm}template$<$typename typ $>$ int {\bf Kolejka}$<$ typ $>$\+::rozmiar\hspace{0.3cm}{\ttfamily [private]}}\label{class_kolejka_a031fd520e5f22daac808dbeeb705832a}
przechowywane elementy 

Definicja w linii 21 pliku Kolejka.\+hh.

\hypertarget{class_kolejka_a333beaaaf8ccecbfd4b84a7840d3bd33}{}\index{Kolejka@{Kolejka}!tab@{tab}}
\index{tab@{tab}!Kolejka@{Kolejka}}
\paragraph[{tab}]{\setlength{\rightskip}{0pt plus 5cm}template$<$typename typ $>$ typ$\ast$ {\bf Kolejka}$<$ typ $>$\+::tab\hspace{0.3cm}{\ttfamily [private]}}\label{class_kolejka_a333beaaaf8ccecbfd4b84a7840d3bd33}
koniec 

Definicja w linii 20 pliku Kolejka.\+hh.



Dokumentacja dla tej klasy została wygenerowana z pliku\+:\begin{DoxyCompactItemize}
\item 
\hyperlink{_kolejka_8hh}{Kolejka.\+hh}\end{DoxyCompactItemize}

\hypertarget{class_lista}{}\subsection{Dokumentacja szablonu klasy Lista$<$ typ $>$}
\label{class_lista}\index{Lista$<$ typ $>$@{Lista$<$ typ $>$}}


definicja klasy \hyperlink{class_lista}{Lista} Przechowuje obiekt oraz wskaznik na nastepny i pole rozmiar. Zbudowana na szablonie.  




{\ttfamily \#include $<$Lista.\+hh$>$}

\subsubsection*{Metody publiczne}
\begin{DoxyCompactItemize}
\item 
\hyperlink{class_lista_a23a5b3313a893057276942e74f330b89}{Lista} ()
\begin{DoxyCompactList}\small\item\em definicja konstruktora bezparametrycznego Zeruje rozmiar, ustawia wskaznik na N\+U\+L\+L. \end{DoxyCompactList}\item 
\hyperlink{class_lista_accc5a3585c7f97372f35f81ef574646e}{$\sim$\+Lista} ()
\begin{DoxyCompactList}\small\item\em definicja destruktora Zeruje rozmiar, Kasuje wszystkie obiektry/elementy. \end{DoxyCompactList}\item 
void \hyperlink{class_lista_afe3d2ff8a9161d30301bfc210287ab51}{push} (typ element)
\begin{DoxyCompactList}\small\item\em definicja metody push \end{DoxyCompactList}\item 
typ \hyperlink{class_lista_a536acf0c981ac359d145da0cc452aadf}{pop} ()
\begin{DoxyCompactList}\small\item\em definicja metody pop \end{DoxyCompactList}\item 
int \hyperlink{class_lista_a8025f28bcc402832d854af6b874451f1}{size} () const 
\begin{DoxyCompactList}\small\item\em deinicja metody size \end{DoxyCompactList}\end{DoxyCompactItemize}
\subsubsection*{Atrybuty prywatne}
\begin{DoxyCompactItemize}
\item 
\hyperlink{class_lista}{Lista}$<$ typ $>$ $\ast$ \hyperlink{class_lista_a42fc5a822a0f442758c88cbf1e19d15f}{nastepny}
\item 
typ \hyperlink{class_lista_a017ecd407cac7e93841f69280f4caa1a}{dane}
\item 
int \hyperlink{class_lista_a5b9d349b6ba27b55058ca77cb0909b6e}{rozmiar}
\end{DoxyCompactItemize}


\subsubsection{Opis szczegółowy}
\subsubsection*{template$<$typename typ$>$class Lista$<$ typ $>$}

definicja klasy \hyperlink{class_lista}{Lista} Przechowuje obiekt oraz wskaznik na nastepny i pole rozmiar. Zbudowana na szablonie. 

Definicja w linii 15 pliku Lista.\+hh.



\subsubsection{Dokumentacja konstruktora i destruktora}
\hypertarget{class_lista_a23a5b3313a893057276942e74f330b89}{}\index{Lista@{Lista}!Lista@{Lista}}
\index{Lista@{Lista}!Lista@{Lista}}
\paragraph[{Lista}]{\setlength{\rightskip}{0pt plus 5cm}template$<$typename typ $>$ {\bf Lista}$<$ typ $>$\+::{\bf Lista} (
\begin{DoxyParamCaption}
{}
\end{DoxyParamCaption}
)}\label{class_lista_a23a5b3313a893057276942e74f330b89}


definicja konstruktora bezparametrycznego Zeruje rozmiar, ustawia wskaznik na N\+U\+L\+L. 

ilosc elementow/obiektow 

Definicja w linii 33 pliku Lista.\+hh.

\hypertarget{class_lista_accc5a3585c7f97372f35f81ef574646e}{}\index{Lista@{Lista}!````~Lista@{$\sim$\+Lista}}
\index{````~Lista@{$\sim$\+Lista}!Lista@{Lista}}
\paragraph[{$\sim$\+Lista}]{\setlength{\rightskip}{0pt plus 5cm}template$<$typename typ $>$ {\bf Lista}$<$ typ $>$\+::$\sim${\bf Lista} (
\begin{DoxyParamCaption}
{}
\end{DoxyParamCaption}
)}\label{class_lista_accc5a3585c7f97372f35f81ef574646e}


definicja destruktora Zeruje rozmiar, Kasuje wszystkie obiektry/elementy. 



Definicja w linii 94 pliku Lista.\+hh.



\subsubsection{Dokumentacja funkcji składowych}
\hypertarget{class_lista_a536acf0c981ac359d145da0cc452aadf}{}\index{Lista@{Lista}!pop@{pop}}
\index{pop@{pop}!Lista@{Lista}}
\paragraph[{pop}]{\setlength{\rightskip}{0pt plus 5cm}template$<$typename typ $>$ typ {\bf Lista}$<$ typ $>$\+::pop (
\begin{DoxyParamCaption}
{}
\end{DoxyParamCaption}
)}\label{class_lista_a536acf0c981ac359d145da0cc452aadf}


definicja metody pop 

\begin{DoxyReturn}{Zwraca}
usuwany element Ustawia wskaznik na poprzedni element zwraca i kasuje ostatni element. 

0 i wyswietla komunikat gdy lista jest pusta. 
\end{DoxyReturn}


Definicja w linii 62 pliku Lista.\+hh.

\hypertarget{class_lista_afe3d2ff8a9161d30301bfc210287ab51}{}\index{Lista@{Lista}!push@{push}}
\index{push@{push}!Lista@{Lista}}
\paragraph[{push}]{\setlength{\rightskip}{0pt plus 5cm}template$<$typename typ $>$ void {\bf Lista}$<$ typ $>$\+::push (
\begin{DoxyParamCaption}
\item[{typ}]{element}
\end{DoxyParamCaption}
)}\label{class_lista_afe3d2ff8a9161d30301bfc210287ab51}


definicja metody push 


\begin{DoxyParams}{Parametry}
{\em \mbox{[}element\mbox{]}} & dodawany element na koniec listy Zwieksza rozmiar, alokuje pamiec, przypisuje element do pola klasy. \\
\hline
\end{DoxyParams}


Definicja w linii 45 pliku Lista.\+hh.

\hypertarget{class_lista_a8025f28bcc402832d854af6b874451f1}{}\index{Lista@{Lista}!size@{size}}
\index{size@{size}!Lista@{Lista}}
\paragraph[{size}]{\setlength{\rightskip}{0pt plus 5cm}template$<$typename typ $>$ int {\bf Lista}$<$ typ $>$\+::size (
\begin{DoxyParamCaption}
{}
\end{DoxyParamCaption}
) const}\label{class_lista_a8025f28bcc402832d854af6b874451f1}


deinicja metody size 

\begin{DoxyReturn}{Zwraca}
ilosc elementow przechowywanych na liscie. 
\end{DoxyReturn}


Definicja w linii 83 pliku Lista.\+hh.



\subsubsection{Dokumentacja atrybutów składowych}
\hypertarget{class_lista_a017ecd407cac7e93841f69280f4caa1a}{}\index{Lista@{Lista}!dane@{dane}}
\index{dane@{dane}!Lista@{Lista}}
\paragraph[{dane}]{\setlength{\rightskip}{0pt plus 5cm}template$<$typename typ$>$ typ {\bf Lista}$<$ typ $>$\+::dane\hspace{0.3cm}{\ttfamily [private]}}\label{class_lista_a017ecd407cac7e93841f69280f4caa1a}
wskaznik na nastepny obiekt/element 

Definicja w linii 17 pliku Lista.\+hh.

\hypertarget{class_lista_a42fc5a822a0f442758c88cbf1e19d15f}{}\index{Lista@{Lista}!nastepny@{nastepny}}
\index{nastepny@{nastepny}!Lista@{Lista}}
\paragraph[{nastepny}]{\setlength{\rightskip}{0pt plus 5cm}template$<$typename typ$>$ {\bf Lista}$<$typ$>$$\ast$ {\bf Lista}$<$ typ $>$\+::nastepny\hspace{0.3cm}{\ttfamily [private]}}\label{class_lista_a42fc5a822a0f442758c88cbf1e19d15f}


Definicja w linii 16 pliku Lista.\+hh.

\hypertarget{class_lista_a5b9d349b6ba27b55058ca77cb0909b6e}{}\index{Lista@{Lista}!rozmiar@{rozmiar}}
\index{rozmiar@{rozmiar}!Lista@{Lista}}
\paragraph[{rozmiar}]{\setlength{\rightskip}{0pt plus 5cm}template$<$typename typ$>$ int {\bf Lista}$<$ typ $>$\+::rozmiar\hspace{0.3cm}{\ttfamily [private]}}\label{class_lista_a5b9d349b6ba27b55058ca77cb0909b6e}
przechowywana informacja/obiekt/element 

Definicja w linii 18 pliku Lista.\+hh.


\hypertarget{class_stos}{}\subsection{Dokumentacja szablonu klasy Stos$<$ typ $>$}
\label{class_stos}\index{Stos$<$ typ $>$@{Stos$<$ typ $>$}}


definicja klasy \hyperlink{class_stos}{Stos} zedfiniowany za pomoca tablicy. Klasa zbudowana na szablonie.  




{\ttfamily \#include $<$Stos.\+hh$>$}

\subsubsection*{Metody publiczne}
\begin{DoxyCompactItemize}
\item 
\hyperlink{class_stos_af6c53f2458ebd95fd992a14fef6712d0}{Stos} (typ p)
\begin{DoxyCompactList}\small\item\em definicja konstruktora z jednym parametrem \end{DoxyCompactList}\item 
\hyperlink{class_stos_afc525fb8a9f8f80fda9bf0f846c078c4}{Stos} ()
\begin{DoxyCompactList}\small\item\em definicja konstruktora bezparametrycznego zeruje rozmiar, przypisuje N\+U\+L\+L do wskaznikow. \end{DoxyCompactList}\item 
\hyperlink{class_stos_a1e9ba3ec6f2759c1fd9a69b56b0d0c5f}{$\sim$\+Stos} ()
\begin{DoxyCompactList}\small\item\em definicja destruktora Zwalnia pamiec, zeruje rozmiar. \end{DoxyCompactList}\item 
void \hyperlink{class_stos_a51e06002fe60a5946cb2dd82cd0c0e06}{push} (typ element)
\begin{DoxyCompactList}\small\item\em definicja metody push \end{DoxyCompactList}\item 
typ \hyperlink{class_stos_a779da9dd1daf1118cd4b213401a4e1ed}{pop} ()
\begin{DoxyCompactList}\small\item\em definicja metody pop zmmniejsza rozmiar o 1, \end{DoxyCompactList}\item 
int \hyperlink{class_stos_aedf6a4f76e7f0bce073a21be4db7163c}{size} () const 
\begin{DoxyCompactList}\small\item\em definicja metody size \end{DoxyCompactList}\end{DoxyCompactItemize}
\subsubsection*{Atrybuty prywatne}
\begin{DoxyCompactItemize}
\item 
int \hyperlink{class_stos_a66c92dc47edd280d9ea15fbcafcd9a80}{rozmiar}
\item 
typ $\ast$ \hyperlink{class_stos_acb9c6baeb0616796d20cfd05f6457fe3}{tab}
\end{DoxyCompactItemize}


\subsubsection{Opis szczegółowy}
\subsubsection*{template$<$typename typ$>$class Stos$<$ typ $>$}

definicja klasy \hyperlink{class_stos}{Stos} zedfiniowany za pomoca tablicy. Klasa zbudowana na szablonie. 

Definicja w linii 17 pliku Stos.\+hh.



\subsubsection{Dokumentacja konstruktora i destruktora}
\hypertarget{class_stos_af6c53f2458ebd95fd992a14fef6712d0}{}\index{Stos@{Stos}!Stos@{Stos}}
\index{Stos@{Stos}!Stos@{Stos}}
\paragraph[{Stos}]{\setlength{\rightskip}{0pt plus 5cm}template$<$typename typ $>$ {\bf Stos}$<$ typ $>$\+::{\bf Stos} (
\begin{DoxyParamCaption}
\item[{typ}]{p}
\end{DoxyParamCaption}
)}\label{class_stos_af6c53f2458ebd95fd992a14fef6712d0}


definicja konstruktora z jednym parametrem 

alokowana pamiec


\begin{DoxyParams}{Parametry}
{\em \mbox{[}p\mbox{]}} & rozmiar ilosci alokowanej pamieci alokuje pamiec o zadanym rozmiarze \\
\hline
\end{DoxyParams}


Definicja w linii 36 pliku Stos.\+hh.

\hypertarget{class_stos_afc525fb8a9f8f80fda9bf0f846c078c4}{}\index{Stos@{Stos}!Stos@{Stos}}
\index{Stos@{Stos}!Stos@{Stos}}
\paragraph[{Stos}]{\setlength{\rightskip}{0pt plus 5cm}template$<$typename typ $>$ {\bf Stos}$<$ typ $>$\+::{\bf Stos} (
\begin{DoxyParamCaption}
{}
\end{DoxyParamCaption}
)}\label{class_stos_afc525fb8a9f8f80fda9bf0f846c078c4}


definicja konstruktora bezparametrycznego zeruje rozmiar, przypisuje N\+U\+L\+L do wskaznikow. 



Definicja w linii 47 pliku Stos.\+hh.

\hypertarget{class_stos_a1e9ba3ec6f2759c1fd9a69b56b0d0c5f}{}\index{Stos@{Stos}!````~Stos@{$\sim$\+Stos}}
\index{````~Stos@{$\sim$\+Stos}!Stos@{Stos}}
\paragraph[{$\sim$\+Stos}]{\setlength{\rightskip}{0pt plus 5cm}template$<$typename typ $>$ {\bf Stos}$<$ typ $>$\+::$\sim${\bf Stos} (
\begin{DoxyParamCaption}
{}
\end{DoxyParamCaption}
)}\label{class_stos_a1e9ba3ec6f2759c1fd9a69b56b0d0c5f}


definicja destruktora Zwalnia pamiec, zeruje rozmiar. 



Definicja w linii 58 pliku Stos.\+hh.



\subsubsection{Dokumentacja funkcji składowych}
\hypertarget{class_stos_a779da9dd1daf1118cd4b213401a4e1ed}{}\index{Stos@{Stos}!pop@{pop}}
\index{pop@{pop}!Stos@{Stos}}
\paragraph[{pop}]{\setlength{\rightskip}{0pt plus 5cm}template$<$typename typ $>$ typ {\bf Stos}$<$ typ $>$\+::pop (
\begin{DoxyParamCaption}
{}
\end{DoxyParamCaption}
)}\label{class_stos_a779da9dd1daf1118cd4b213401a4e1ed}


definicja metody pop zmmniejsza rozmiar o 1, 

\begin{DoxyReturn}{Zwraca}
usuwany element 

0 i wyswietla komunikat, kiedy stos jest pusty. 
\end{DoxyReturn}


Definicja w linii 92 pliku Stos.\+hh.

\hypertarget{class_stos_a51e06002fe60a5946cb2dd82cd0c0e06}{}\index{Stos@{Stos}!push@{push}}
\index{push@{push}!Stos@{Stos}}
\paragraph[{push}]{\setlength{\rightskip}{0pt plus 5cm}template$<$typename typ $>$ void {\bf Stos}$<$ typ $>$\+::push (
\begin{DoxyParamCaption}
\item[{typ}]{element}
\end{DoxyParamCaption}
)}\label{class_stos_a51e06002fe60a5946cb2dd82cd0c0e06}


definicja metody push 


\begin{DoxyParams}{Parametry}
{\em \mbox{[}element\mbox{]}} & dodany element na koniec stosu zwieksza rozmiar o 1, alokuje nowa tablice, kopiuje zawartosc starej do nowej, kladzie element na ostatniej pozycji, realokuje i przekopiowuje zawartosc do pierwotenj tablicy, usuwa tablice pomocnicza. \\
\hline
\end{DoxyParams}


Definicja w linii 73 pliku Stos.\+hh.

\hypertarget{class_stos_aedf6a4f76e7f0bce073a21be4db7163c}{}\index{Stos@{Stos}!size@{size}}
\index{size@{size}!Stos@{Stos}}
\paragraph[{size}]{\setlength{\rightskip}{0pt plus 5cm}template$<$typename typ $>$ int {\bf Stos}$<$ typ $>$\+::size (
\begin{DoxyParamCaption}
{}
\end{DoxyParamCaption}
) const}\label{class_stos_aedf6a4f76e7f0bce073a21be4db7163c}


definicja metody size 

\begin{DoxyReturn}{Zwraca}
rozmiar stosu 
\end{DoxyReturn}


Definicja w linii 111 pliku Stos.\+hh.



\subsubsection{Dokumentacja atrybutów składowych}
\hypertarget{class_stos_a66c92dc47edd280d9ea15fbcafcd9a80}{}\index{Stos@{Stos}!rozmiar@{rozmiar}}
\index{rozmiar@{rozmiar}!Stos@{Stos}}
\paragraph[{rozmiar}]{\setlength{\rightskip}{0pt plus 5cm}template$<$typename typ$>$ int {\bf Stos}$<$ typ $>$\+::rozmiar\hspace{0.3cm}{\ttfamily [private]}}\label{class_stos_a66c92dc47edd280d9ea15fbcafcd9a80}


Definicja w linii 18 pliku Stos.\+hh.

\hypertarget{class_stos_acb9c6baeb0616796d20cfd05f6457fe3}{}\index{Stos@{Stos}!tab@{tab}}
\index{tab@{tab}!Stos@{Stos}}
\paragraph[{tab}]{\setlength{\rightskip}{0pt plus 5cm}template$<$typename typ$>$ typ$\ast$ {\bf Stos}$<$ typ $>$\+::tab\hspace{0.3cm}{\ttfamily [private]}}\label{class_stos_acb9c6baeb0616796d20cfd05f6457fe3}
rozmiar stosu 

Definicja w linii 19 pliku Stos.\+hh.


\section{Dokumentacja plików}
\hypertarget{_kolejka_8hh}{}\subsection{Dokumentacja pliku Kolejka.\+hh}
\label{_kolejka_8hh}\index{Kolejka.\+hh@{Kolejka.\+hh}}


definicja struktury danych \hyperlink{class_kolejka}{Kolejka}  


Ten wykres pokazuje, które pliki bezpośrednio lub pośrednio załączają ten plik\+:

\hypertarget{_lista_8hh}{}\subsection{Dokumentacja pliku Lista.\+hh}
\label{_lista_8hh}\index{Lista.\+hh@{Lista.\+hh}}


definicja struktury danych \hyperlink{class_lista}{Lista}  


Ten wykres pokazuje, które pliki bezpośrednio lub pośrednio załączają ten plik\+:
% FIG 0
\subsubsection*{Komponenty}
\begin{DoxyCompactItemize}
\item 
class \hyperlink{class_lista}{Lista$<$ typ $>$}
\begin{DoxyCompactList}\small\item\em definicja klasy \hyperlink{class_lista}{Lista} Przechowuje obiekt oraz wskaznik na nastepny i pole rozmiar. Zbudowana na szablonie. \end{DoxyCompactList}\end{DoxyCompactItemize}

\hypertarget{main_8cpp}{}\subsection{main.\+cpp File Reference}
\label{main_8cpp}\index{main.\+cpp@{main.\+cpp}}
{\ttfamily \#include $<$iostream$>$}\\*
{\ttfamily \#include $<$ctime$>$}\\*
{\ttfamily \#include \char`\"{}list.\+hh\char`\"{}}\\*
{\ttfamily \#include \char`\"{}benchmark.\+hh\char`\"{}}\\*
{\ttfamily \#include \char`\"{}csort.\+hh\char`\"{}}\\*
{\ttfamily \#include \char`\"{}heap\+\_\+sort.\+hh\char`\"{}}\\*
{\ttfamily \#include \char`\"{}quick\+\_\+sort.\+hh\char`\"{}}\\*
{\ttfamily \#include \char`\"{}merge\+\_\+sort.\+hh\char`\"{}}\\*
\subsubsection*{Functions}
\begin{DoxyCompactItemize}
\item 
int \hyperlink{main_8cpp_ae66f6b31b5ad750f1fe042a706a4e3d4}{main} ()
\end{DoxyCompactItemize}


\subsubsection{Function Documentation}
\hypertarget{main_8cpp_ae66f6b31b5ad750f1fe042a706a4e3d4}{}\index{main.\+cpp@{main.\+cpp}!main@{main}}
\index{main@{main}!main.\+cpp@{main.\+cpp}}
\paragraph[{main}]{\setlength{\rightskip}{0pt plus 5cm}int main (
\begin{DoxyParamCaption}
{}
\end{DoxyParamCaption}
)}\label{main_8cpp_ae66f6b31b5ad750f1fe042a706a4e3d4}

\hypertarget{_stoper_8hh}{}\subsection{Dokumentacja pliku Stoper.\+hh}
\label{_stoper_8hh}\index{Stoper.\+hh@{Stoper.\+hh}}


definicje funkcji zliczajacych czas operacji wypelnienia struktur danych  


{\ttfamily \#include $<$windows.\+h$>$}\\*
{\ttfamily \#include $<$ctime$>$}\\*
{\ttfamily \#include $<$fstream$>$}\\*
Wykres zależności załączania dla Stoper.\+hh\+:
% FIG 0
Ten wykres pokazuje, które pliki bezpośrednio lub pośrednio załączają ten plik\+:
% FIG 1
\subsubsection*{Funkcje}
\begin{DoxyCompactItemize}
\item 
L\+A\+R\+G\+E\+\_\+\+I\+N\+T\+E\+G\+E\+R \hyperlink{_stoper_8hh_a4fbb2a5fceb3efd577419fc87c6f3d07}{start\+Timer} ()
\begin{DoxyCompactList}\small\item\em definicja funkcji Start\+Timer Rozpoczyna pomiar czasu Funkcja pobrana ze strony \href{http://jaroslaw.mierzwa.staff.iiar.pwr.wroc.pl/}{\tt http\+://jaroslaw.\+mierzwa.\+staff.\+iiar.\+pwr.\+wroc.\+pl/} \end{DoxyCompactList}\item 
L\+A\+R\+G\+E\+\_\+\+I\+N\+T\+E\+G\+E\+R \hyperlink{_stoper_8hh_abd320a48742faf3aa6ad1a7e7c6b518b}{end\+Timer} ()
\begin{DoxyCompactList}\small\item\em definicja funkcji end\+Timer Konczy pomiar czasu Funkcja pobrana ze strony \href{http://jaroslaw.mierzwa.staff.iiar.pwr.wroc.pl/}{\tt http\+://jaroslaw.\+mierzwa.\+staff.\+iiar.\+pwr.\+wroc.\+pl/} \end{DoxyCompactList}\item 
{\footnotesize template$<$typename typ $>$ }\\double \hyperlink{_stoper_8hh_a4fac2b3b7d9ad98fc63291c183cde774}{licz} (typ obiekt, int N)
\begin{DoxyCompactList}\small\item\em definicja funkcji benchamarkujacej licz funkcja nie przeznaczona dla struktury danych \hyperlink{class_kolejka}{Kolejka} \end{DoxyCompactList}\item 
{\footnotesize template$<$typename typ $>$ }\\double \hyperlink{_stoper_8hh_acbfcf446d0eef3844d061cfc2c5b2f24}{licz200} (typ obiekt, int N)
\begin{DoxyCompactList}\small\item\em definicja funkcji benchamarkujacej licz200 Funkcja przeznaczonadla struktury danych stos, dzialajacej przy alokowaniu pamieci razy 200\%. Funkcja nie przeznaczona dla struktury danych \hyperlink{class_kolejka}{Kolejka} i lista oraz stos dzialajacej w sposub alokowania pamieci o 1 element. \end{DoxyCompactList}\item 
{\footnotesize template$<$typename typ $>$ }\\double \hyperlink{_stoper_8hh_ae0470166e6ffb50297e95736e5cb3ee2}{licz\+Kol} (typ objekt, int N)
\begin{DoxyCompactList}\small\item\em definicja funkcji licz\+Kol Funkcja Benchmarkujaca przeznaczona tylko dla struktury danych \hyperlink{class_kolejka}{Kolejka} Posiada atrybuty takie same jak funkcja licz \end{DoxyCompactList}\item 
{\footnotesize template$<$typename typ $>$ }\\double \hyperlink{_stoper_8hh_aa86704719385cc3fd8e95d656090445e}{zliczaj} (typ obiekt, int N)
\begin{DoxyCompactList}\small\item\em definicja funkcji zliczaj Alternatywna funkcja benchamrkujaca nie przeznaczona dla struktury danych \hyperlink{class_kolejka}{Kolejka} \end{DoxyCompactList}\end{DoxyCompactItemize}


\subsubsection{Dokumentacja funkcji}
\hypertarget{_stoper_8hh_abd320a48742faf3aa6ad1a7e7c6b518b}{}\index{Stoper.\+hh@{Stoper.\+hh}!end\+Timer@{end\+Timer}}
\index{end\+Timer@{end\+Timer}!Stoper.\+hh@{Stoper.\+hh}}
\paragraph[{end\+Timer}]{\setlength{\rightskip}{0pt plus 5cm}L\+A\+R\+G\+E\+\_\+\+I\+N\+T\+E\+G\+E\+R end\+Timer (
\begin{DoxyParamCaption}
{}
\end{DoxyParamCaption}
)}\label{_stoper_8hh_abd320a48742faf3aa6ad1a7e7c6b518b}


Definicja w linii 31 pliku Stoper.\+hh.

\hypertarget{_stoper_8hh_a4fac2b3b7d9ad98fc63291c183cde774}{}\index{Stoper.\+hh@{Stoper.\+hh}!licz@{licz}}
\index{licz@{licz}!Stoper.\+hh@{Stoper.\+hh}}
\paragraph[{licz}]{\setlength{\rightskip}{0pt plus 5cm}template$<$typename typ $>$ double licz (
\begin{DoxyParamCaption}
\item[{typ}]{obiekt, }
\item[{int}]{N}
\end{DoxyParamCaption}
)}\label{_stoper_8hh_a4fac2b3b7d9ad98fc63291c183cde774}

\begin{DoxyParams}{Parametry}
{\em \mbox{[}obiekt\mbox{]}} & struktura danych, dla ktorej zostana przeprowadzone obliczenia \\
\hline
{\em \mbox{[}\+N\mbox{]}} & ilosc elementow, ktorymi zostanie wypelniona strukutra danych \\
\hline
\end{DoxyParams}
\begin{DoxyReturn}{Zwraca}
czas wypelnienia struktury danych liczbami pseldolosowymi Funkcja pobrana ze strony \href{http://jaroslaw.mierzwa.staff.iiar.pwr.wroc.pl/}{\tt http\+://jaroslaw.\+mierzwa.\+staff.\+iiar.\+pwr.\+wroc.\+pl/} 
\end{DoxyReturn}


Definicja w linii 49 pliku Stoper.\+hh.

\hypertarget{_stoper_8hh_acbfcf446d0eef3844d061cfc2c5b2f24}{}\index{Stoper.\+hh@{Stoper.\+hh}!licz200@{licz200}}
\index{licz200@{licz200}!Stoper.\+hh@{Stoper.\+hh}}
\paragraph[{licz200}]{\setlength{\rightskip}{0pt plus 5cm}template$<$typename typ $>$ double licz200 (
\begin{DoxyParamCaption}
\item[{typ}]{obiekt, }
\item[{int}]{N}
\end{DoxyParamCaption}
)}\label{_stoper_8hh_acbfcf446d0eef3844d061cfc2c5b2f24}

\begin{DoxyParams}{Parametry}
{\em \mbox{[}obiekt\mbox{]}} & struktura danych, dla ktorej zostana przeprowadzone obliczenia \\
\hline
{\em \mbox{[}\+N\mbox{]}} & ilosc elementow, ktorymi zostanie wypelniona strukutra danych \\
\hline
\end{DoxyParams}
\begin{DoxyReturn}{Zwraca}
czas wypelnienia struktury danych liczbami pseldolosowymi Funkcja pobrana ze strony \href{http://jaroslaw.mierzwa.staff.iiar.pwr.wroc.pl/}{\tt http\+://jaroslaw.\+mierzwa.\+staff.\+iiar.\+pwr.\+wroc.\+pl/} 
\end{DoxyReturn}


Definicja w linii 76 pliku Stoper.\+hh.

\hypertarget{_stoper_8hh_ae0470166e6ffb50297e95736e5cb3ee2}{}\index{Stoper.\+hh@{Stoper.\+hh}!licz\+Kol@{licz\+Kol}}
\index{licz\+Kol@{licz\+Kol}!Stoper.\+hh@{Stoper.\+hh}}
\paragraph[{licz\+Kol}]{\setlength{\rightskip}{0pt plus 5cm}template$<$typename typ $>$ double licz\+Kol (
\begin{DoxyParamCaption}
\item[{typ}]{objekt, }
\item[{int}]{N}
\end{DoxyParamCaption}
)}\label{_stoper_8hh_ae0470166e6ffb50297e95736e5cb3ee2}


Definicja w linii 97 pliku Stoper.\+hh.

\hypertarget{_stoper_8hh_a4fbb2a5fceb3efd577419fc87c6f3d07}{}\index{Stoper.\+hh@{Stoper.\+hh}!start\+Timer@{start\+Timer}}
\index{start\+Timer@{start\+Timer}!Stoper.\+hh@{Stoper.\+hh}}
\paragraph[{start\+Timer}]{\setlength{\rightskip}{0pt plus 5cm}L\+A\+R\+G\+E\+\_\+\+I\+N\+T\+E\+G\+E\+R start\+Timer (
\begin{DoxyParamCaption}
{}
\end{DoxyParamCaption}
)}\label{_stoper_8hh_a4fbb2a5fceb3efd577419fc87c6f3d07}


Definicja w linii 17 pliku Stoper.\+hh.

\hypertarget{_stoper_8hh_aa86704719385cc3fd8e95d656090445e}{}\index{Stoper.\+hh@{Stoper.\+hh}!zliczaj@{zliczaj}}
\index{zliczaj@{zliczaj}!Stoper.\+hh@{Stoper.\+hh}}
\paragraph[{zliczaj}]{\setlength{\rightskip}{0pt plus 5cm}template$<$typename typ $>$ double zliczaj (
\begin{DoxyParamCaption}
\item[{typ}]{obiekt, }
\item[{int}]{N}
\end{DoxyParamCaption}
)}\label{_stoper_8hh_aa86704719385cc3fd8e95d656090445e}

\begin{DoxyParams}{Parametry}
{\em \mbox{[}obiekt\mbox{]}} & struktura danych \\
\hline
{\em \mbox{[}\+N\mbox{]}} & ilosc liczb pseldolosowych, ktorymi zostanie wypelniona struktura danych \\
\hline
\end{DoxyParams}
\begin{DoxyReturn}{Zwraca}
czas wykonania operacji podany w ms. Funkcja nie wykorzystana w programie!! 
\end{DoxyReturn}


Definicja w linii 122 pliku Stoper.\+hh.


\hypertarget{_stos_8hh}{}\subsection{Dokumentacja pliku Stos.\+hh}
\label{_stos_8hh}\index{Stos.\+hh@{Stos.\+hh}}


definicja struktruy danych \hyperlink{class_stos}{Stos}  


Ten wykres pokazuje, które pliki bezpośrednio lub pośrednio załączają ten plik\+:
% FIG 0
\subsubsection*{Komponenty}
\begin{DoxyCompactItemize}
\item 
class \hyperlink{class_stos}{Stos$<$ typ $>$}
\begin{DoxyCompactList}\small\item\em definicja klasy \hyperlink{class_stos}{Stos} zedfiniowany za pomoca tablicy. Klasa zbudowana na szablonie. \end{DoxyCompactList}\end{DoxyCompactItemize}



\title{Laboratorium 3 - Sprawozdanie}
\author{Wojciech Makuch}

\maketitle
\section{Zadanie}
Program framework benchmarkujacy dla zaimplentowanej struktury danych Stos.
\section{Realizacja}
Program zawiera 3 struktury danych. Każda z nich zawiera 3 podstawowe metody: połóż element, zdejmij element, zwróć rozmiar. Struktua Stos zbudowana za pomocą tablicy z realokacją pamięci, Lista ze wskaźnikiem na następny element, oraz Kolejka z indeksami na pierwszy i ostatni element. Wszystkie struktury danych działają prawidłowo.
Ponadto program zawiera funkcje wypęłniającą struktury liczbami psełdolosowymi oraz zliczającą czas dla przeprowadzenia testów złożonosci obliczeniowej ww. struktur.
\section{Działanie}
Głowna funkcja programu działa tylko na strukturze typu Stos. Testuje jego 2 metody push() oraz push200(). Metoda push() polega na powiekszaniu alokowanej pamieci o 1 element, natomiast push200() polega na alokowaniu pamieci razy 200\%. Działanie programu polega na zliczeniu czasu wypelniania tej struktury liczbami pseudolosowymi oraz zapisania wyników do pliku o nazwie \textsl{Pomiar\_czasu2.txt.}
\section{Wyniki}
Podczas alokowania pamieci struktura typu stos obsługiwana przez metode push() może alokować pamięć na maksimum $10^{5}$ elementów. Natomiast dzięki metodzie push2000() rozmiar ten zostaje zwiększony do $10^{7}$. Ponadto można zauważyć dłuższy czas destrukcji struktury zaalokowanej przez push().Z danych dostarczonych przez program wynika, że metoda push200() działa o wiele szybciej i jest w stanie zaalokować więcej pamięci. Na Rys 1. pokazano w skali logarytmicznej wykres zależności ilości elementów od czasu potrzebnego na wypełnienie nimi struktury.
\begin{figure}[h!]
\centering
\includegraphics[scale=0.7]{wykres1}
\caption{Wykres złożoności obliczeniowej}
\label{fig:wykres1}
\end{figure}
Z wykresu można zauważyć, że złożoność obliczeniowa jest w przybliżeniu liniowa, czyli O(n). Ponadto złożoność metody push() rośnie o wiele szybciej, co znaczy, że program ma gorszą złożonoś obliczeniową. Dodatkowo na wykresie zaznaczono dla porównania przebieg idealnej funkcji liniowej. Widać, że metoda push200() ma nawet lepszą złożoność niż O(n). :)
\section{Komentarz}
Do utworzenia dokumentacji wykorzystano system Doxygen.
Funkcja pomiaru czasu dla systemu Windows pobrana ze strony dr. J. Mierzwy. Program skompilowano w środowisku Code::Blocks. Do stworzonia wykresu posłużono się pakietem MS Excel, sprawozdanie napisano używając systemu \LaTeX.
\end{document}