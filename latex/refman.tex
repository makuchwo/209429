\documentclass[twoside]{article}

% Packages required by doxygen
\usepackage{fixltx2e}
\usepackage{calc}
\usepackage{doxygen}
\usepackage[export]{adjustbox} % also loads graphicx
\usepackage{graphicx}
\usepackage[utf8]{inputenc}
\usepackage{makeidx}
\usepackage{multicol}
\usepackage{multirow}
\PassOptionsToPackage{warn}{textcomp}
\usepackage{textcomp}
\usepackage[nointegrals]{wasysym}
\usepackage[table]{xcolor}

% Font selection
\usepackage[T1]{fontenc}
\usepackage[scaled=.90]{helvet}
\usepackage{courier}
\usepackage{amssymb}
\usepackage{sectsty}
\renewcommand{\familydefault}{\sfdefault}
\allsectionsfont{%
  \fontseries{bc}\selectfont%
  \color{darkgray}%
}
\renewcommand{\DoxyLabelFont}{%
  \fontseries{bc}\selectfont%
  \color{darkgray}%
}
\newcommand{\+}{\discretionary{\mbox{\scriptsize$\hookleftarrow$}}{}{}}

% Page & text layout
\usepackage{geometry}
\geometry{%
  a4paper,%
  top=2.5cm,%
  bottom=2.5cm,%
  left=2.5cm,%
  right=2.5cm%
}
\tolerance=750
\hfuzz=15pt
\hbadness=750
\setlength{\emergencystretch}{15pt}
\setlength{\parindent}{0cm}
\setlength{\parskip}{0.2cm}
\makeatletter
\renewcommand{\paragraph}{%
  \@startsection{paragraph}{4}{0ex}{-1.0ex}{1.0ex}{%
    \normalfont\normalsize\bfseries\SS@parafont%
  }%
}
\renewcommand{\subparagraph}{%
  \@startsection{subparagraph}{5}{0ex}{-1.0ex}{1.0ex}{%
    \normalfont\normalsize\bfseries\SS@subparafont%
  }%
}
\makeatother

% Headers & footers
\usepackage{fancyhdr}
\pagestyle{fancyplain}
\fancyhead[LE]{\fancyplain{}{\bfseries\thepage}}
\fancyhead[CE]{\fancyplain{}{}}
\fancyhead[RE]{\fancyplain{}{\bfseries\leftmark}}
\fancyhead[LO]{\fancyplain{}{\bfseries\rightmark}}
\fancyhead[CO]{\fancyplain{}{}}
\fancyhead[RO]{\fancyplain{}{\bfseries\thepage}}
\fancyfoot[LE]{\fancyplain{}{}}
\fancyfoot[CE]{\fancyplain{}{}}
\fancyfoot[RE]{\fancyplain{}{\bfseries\scriptsize Generated on Wed May 13 2015 14\+:14\+:37 for Laboratorium 8 -\/ Projektowanie algorytmów i metod sztucznej inteligencji. by Doxygen }}
\fancyfoot[LO]{\fancyplain{}{\bfseries\scriptsize Generated on Wed May 13 2015 14\+:14\+:37 for Laboratorium 8 -\/ Projektowanie algorytmów i metod sztucznej inteligencji. by Doxygen }}
\fancyfoot[CO]{\fancyplain{}{}}
\fancyfoot[RO]{\fancyplain{}{}}
\renewcommand{\footrulewidth}{0.4pt}
\renewcommand{\sectionmark}[1]{%
  \markright{\thesection\ #1}%
}

% Indices & bibliography
\usepackage{natbib}
\usepackage[titles]{tocloft}
\setcounter{tocdepth}{3}
\setcounter{secnumdepth}{5}
\makeindex

% Hyperlinks (required, but should be loaded last)
\usepackage{ifpdf}
\ifpdf
  \usepackage[pdftex,pagebackref=true]{hyperref}
\else
  \usepackage[ps2pdf,pagebackref=true]{hyperref}
\fi
\hypersetup{%
  colorlinks=true,%
  linkcolor=blue,%
  citecolor=blue,%
  unicode%
}

% Custom commands
\newcommand{\clearemptydoublepage}{%
  \newpage{\pagestyle{empty}\cleardoublepage}%
}


%===== C O N T E N T S =====

\begin{document}

% Titlepage & ToC
\hypersetup{pageanchor=false,
             bookmarks=true,
             bookmarksnumbered=true,
             pdfencoding=unicode
            }
\pagenumbering{roman}
\begin{titlepage}
\vspace*{7cm}
\begin{center}%
{\Large Laboratorium 7 -\/ Projektowanie algorytmów i metod sztucznej inteligencji. \\[1ex]\large 2.\+0 }\\
\vspace*{1cm}
{\large Generated by Doxygen 1.8.9.1}\\
\vspace*{0.5cm}
{\small Wed May 13 2015 14:14:37}\\
\end{center}
\end{titlepage}
\tableofcontents
\pagenumbering{arabic}
\hypersetup{pageanchor=true}

%--- Begin generated contents ---
\section{modyfkacja sortowania -\/ benchmark -\/ obserwator}
\label{index}\hypertarget{index}{}\begin{DoxyAuthor}{Autor}
Wojciech Makuch 
\end{DoxyAuthor}
\begin{DoxyDate}{Data}
27.\+05.\+2015 
\end{DoxyDate}
\begin{DoxyVersion}{Wersja}
1.\+0
\end{DoxyVersion}
program testujacy oraz benchmarkujacy zaimplementowane grafy posiada menu uzytkownika do wyboru \hypertarget{index_utworz}{}\subsection{graf}\label{index_utworz}
\hypertarget{index_utworz}{}\subsection{graf}\label{index_utworz}
\hypertarget{index_wyswietl}{}\subsection{macierz sasiedztwa}\label{index_wyswietl}
\hypertarget{index_DSF}{}\subsection{D\+S\+F}\label{index_DSF}
\hypertarget{index_BFS}{}\subsection{B\+F\+S}\label{index_BFS}
\hypertarget{index_benchmarkuj}{}\subsection{D\+F\+Sa}\label{index_benchmarkuj}
\hypertarget{index_benchmarkuj}{}\subsection{D\+F\+Sa}\label{index_benchmarkuj}
\hypertarget{index_graf}{}\subsection{omawiany na cwiczeniach}\label{index_graf}
\hypertarget{main.cpp_wyjscie}{}\subsection{wyjscie}\label{main.cpp_wyjscie}

\section{Hierarchical Index}
\subsection{Class Hierarchy}
This inheritance list is sorted roughly, but not completely, alphabetically\+:\begin{DoxyCompactList}
\item \contentsline{section}{C\+Benchmark}{\pageref{class_c_benchmark}}{}
\begin{DoxyCompactList}
\item \contentsline{section}{C\+Sort}{\pageref{class_c_sort}}{}
\begin{DoxyCompactList}
\item \contentsline{section}{C\+Heap\+Sort}{\pageref{class_c_heap_sort}}{}
\item \contentsline{section}{C\+Merge\+Sort}{\pageref{class_c_merge_sort}}{}
\item \contentsline{section}{C\+Quick\+Sort}{\pageref{class_c_quick_sort}}{}
\end{DoxyCompactList}
\end{DoxyCompactList}
\item \contentsline{section}{C\+List}{\pageref{class_c_list}}{}
\end{DoxyCompactList}

\section{Class Index}
\subsection{Lista klas}
Tutaj znajdują się klasy, struktury, unie i interfejsy wraz z ich krótkimi opisami\+:\begin{DoxyCompactList}
\item\contentsline{section}{\hyperlink{class_c_benchmark}{C\+Benchmark} }{\pageref{class_c_benchmark}}{}
\item\contentsline{section}{\hyperlink{class_c_edge}{C\+Edge} \\*Definicja klasy \hyperlink{class_c_edge}{C\+Edge} definuje krawedz grafu nie zawiera wag }{\pageref{class_c_edge}}{}
\item\contentsline{section}{\hyperlink{class_c_graph}{C\+Graph} \\*Definicja klasy \hyperlink{class_c_graph}{C\+Graph} definije graf skierowany bez wagowaych krawedzi dziedzicy po klasie \hyperlink{class_c_benchmark}{C\+Benchmark} }{\pageref{class_c_graph}}{}
\item\contentsline{section}{\hyperlink{class_c_node}{C\+Node} \\*Definicja klasy \hyperlink{class_c_node}{C\+Node} definiuje pojedynczy wezel grafu }{\pageref{class_c_node}}{}
\item\contentsline{section}{\hyperlink{classqueue}{queue} \\*Definicja klasy queue definicja kolejki A\+D\+T, kolejka typu F\+I\+F\+O zimplementowaa na liscie }{\pageref{classqueue}}{}
\item\contentsline{section}{\hyperlink{classqueue__node}{queue\+\_\+node} \\*Definicja klasy \hyperlink{classqueue__node}{queue\+\_\+node} definicja wezla dla kolejki definiuje pojedynczy element bedacy w kolejkce }{\pageref{classqueue__node}}{}
\end{DoxyCompactList}

\section{File Index}
\subsection{Lista plików}
Tutaj znajduje się lista wszystkich plików z ich krótkimi opisami\+:\begin{DoxyCompactList}
\item\contentsline{section}{\hyperlink{format_8h}{format.\+h} }{\pageref{format_8h}}{}
\item\contentsline{section}{\hyperlink{global_8h}{global.\+h} }{\pageref{global_8h}}{}
\item\contentsline{section}{\hyperlink{graph_8cpp}{graph.\+cpp} \\*Implementuje zdefiniowana klase grafu }{\pageref{graph_8cpp}}{}
\item\contentsline{section}{\hyperlink{graph_8hh}{graph.\+hh} \\*Zwiera definicje klas \hyperlink{class_c_node}{C\+Node}, \hyperlink{class_c_edge}{C\+Edge}, C\+Grpah \hyperlink{class_c_node}{C\+Node} -\/ wezel grafu \hyperlink{class_c_edge}{C\+Edge} -\/ krawedz grafu \hyperlink{class_c_graph}{C\+Graph} -\/ graf }{\pageref{graph_8hh}}{}
\item\contentsline{section}{\hyperlink{main_8cpp}{main.\+cpp} }{\pageref{main_8cpp}}{}
\item\contentsline{section}{\hyperlink{queue_8cpp}{queue.\+cpp} \\*Implementuje zdefiniowana klase kolejki }{\pageref{queue_8cpp}}{}
\item\contentsline{section}{\hyperlink{queue_8hh}{queue.\+hh} \\*Zawiera definicje klas \hyperlink{classqueue__node}{queue\+\_\+node}, queue \hyperlink{classqueue__node}{queue\+\_\+node} -\/ wezel kolejki queue -\/ kolejka }{\pageref{queue_8hh}}{}
\item\contentsline{section}{\hyperlink{stack_8hh}{stack.\+hh} \\*Definicja struktury danych \hyperlink{class_lista}{Lista} }{\pageref{stack_8hh}}{}
\end{DoxyCompactList}

\section{Class Documentation}
\hypertarget{class_c_benchmark}{}\subsection{Dokumentacja klasy C\+Benchmark}
\label{class_c_benchmark}\index{C\+Benchmark@{C\+Benchmark}}


{\ttfamily \#include $<$benchmark.\+hh$>$}



Dziedziczona przez \hyperlink{class_c_graph}{C\+Graph}.

\subsubsection*{Metody publiczne}
\begin{DoxyCompactItemize}
\item 
virtual void \hyperlink{class_c_benchmark_aa6f22bf0b316b51db8f439c7420a1666}{start\+\_\+timer} ()
\begin{DoxyCompactList}\small\item\em definicja metody start\+\_\+timer rozpoczyna pomiar czasu zapisuje dane do zmiennej performance\+Count\+Start korzysta z metdoy Start\+Timer() \end{DoxyCompactList}\item 
virtual void \hyperlink{class_c_benchmark_a945aaa453776cd11395166b470d11778}{stop\+\_\+timer} ()
\begin{DoxyCompactList}\small\item\em definicja metody stop\+\_\+timer konczy pomiar czasu zapsiuje dane do zmiennej performance\+Count\+End korzysta z metody end\+Timer \end{DoxyCompactList}\item 
virtual int \hyperlink{class_c_benchmark_acb3046f4f9fdff7c17c7633baf41cf36}{put\+\_\+time\+\_\+to\+\_\+file} (int size\+\_\+of\+\_\+list)
\begin{DoxyCompactList}\small\item\em definicja metody put\+\_\+time\+\_\+to\+\_\+file otiwra plik o nazwie \textquotesingle{}timing.\+txt\textquotesingle{} zapisuje do niego ilosc elementow listy oraz czas przeprowadzenia operacji przez klasy obserwowane zamyka plik. \end{DoxyCompactList}\end{DoxyCompactItemize}
\subsubsection*{Metody prywatne}
\begin{DoxyCompactItemize}
\item 
L\+A\+R\+G\+E\+\_\+\+I\+N\+T\+E\+G\+E\+R \hyperlink{class_c_benchmark_a43dd8d9d01499e28f1dc899ea6f3ed97}{start\+Timer} ()
\item 
L\+A\+R\+G\+E\+\_\+\+I\+N\+T\+E\+G\+E\+R \hyperlink{class_c_benchmark_aab174262b346c522bdbc1df87ed93532}{end\+Timer} ()
\end{DoxyCompactItemize}
\subsubsection*{Atrybuty prywatne}
\begin{DoxyCompactItemize}
\item 
L\+A\+R\+G\+E\+\_\+\+I\+N\+T\+E\+G\+E\+R \hyperlink{class_c_benchmark_a54ec1bb4f1e570341e14adfc6312f1b7}{performance\+Count\+Start}
\item 
L\+A\+R\+G\+E\+\_\+\+I\+N\+T\+E\+G\+E\+R \hyperlink{class_c_benchmark_ae72cb7dad8bf32730109776d546af2ad}{performance\+Count\+End}
\end{DoxyCompactItemize}


\subsubsection{Opis szczegółowy}
definicja klasy \hyperlink{class_c_benchmark}{C\+Benchmark} definijue stoper zliczajacy czas wykoania operacji przez inne klasy jest przykladem wzorca obserwatora obserwuje klase C\+Sort i zlicza czas sortowania listy 

Definicja w linii 16 pliku benchmark.\+hh.



\subsubsection{Dokumentacja funkcji składowych}
\hypertarget{class_c_benchmark_aab174262b346c522bdbc1df87ed93532}{}\index{C\+Benchmark@{C\+Benchmark}!end\+Timer@{end\+Timer}}
\index{end\+Timer@{end\+Timer}!C\+Benchmark@{C\+Benchmark}}
\paragraph[{end\+Timer}]{\setlength{\rightskip}{0pt plus 5cm}L\+A\+R\+G\+E\+\_\+\+I\+N\+T\+E\+G\+E\+R C\+Benchmark\+::end\+Timer (
\begin{DoxyParamCaption}
{}
\end{DoxyParamCaption}
)\hspace{0.3cm}{\ttfamily [private]}}\label{class_c_benchmark_aab174262b346c522bdbc1df87ed93532}


Definicja w linii 19 pliku benchmark.\+cpp.

\hypertarget{class_c_benchmark_acb3046f4f9fdff7c17c7633baf41cf36}{}\index{C\+Benchmark@{C\+Benchmark}!put\+\_\+time\+\_\+to\+\_\+file@{put\+\_\+time\+\_\+to\+\_\+file}}
\index{put\+\_\+time\+\_\+to\+\_\+file@{put\+\_\+time\+\_\+to\+\_\+file}!C\+Benchmark@{C\+Benchmark}}
\paragraph[{put\+\_\+time\+\_\+to\+\_\+file}]{\setlength{\rightskip}{0pt plus 5cm}int C\+Benchmark\+::put\+\_\+time\+\_\+to\+\_\+file (
\begin{DoxyParamCaption}
\item[{int}]{size\+\_\+of\+\_\+list}
\end{DoxyParamCaption}
)\hspace{0.3cm}{\ttfamily [virtual]}}\label{class_c_benchmark_acb3046f4f9fdff7c17c7633baf41cf36}

\begin{DoxyParams}{Parametry}
{\em size\+\_\+of\+\_\+list} & -\/ rozmiar listy \\
\hline
\end{DoxyParams}
\begin{DoxyReturn}{Zwraca}
czas przeprowadzenia operacji 

-\/1 w przypadu bledu otwarcia pliku 
\end{DoxyReturn}


Definicja w linii 38 pliku benchmark.\+cpp.

\hypertarget{class_c_benchmark_aa6f22bf0b316b51db8f439c7420a1666}{}\index{C\+Benchmark@{C\+Benchmark}!start\+\_\+timer@{start\+\_\+timer}}
\index{start\+\_\+timer@{start\+\_\+timer}!C\+Benchmark@{C\+Benchmark}}
\paragraph[{start\+\_\+timer}]{\setlength{\rightskip}{0pt plus 5cm}void C\+Benchmark\+::start\+\_\+timer (
\begin{DoxyParamCaption}
{}
\end{DoxyParamCaption}
)\hspace{0.3cm}{\ttfamily [virtual]}}\label{class_c_benchmark_aa6f22bf0b316b51db8f439c7420a1666}


Definicja w linii 28 pliku benchmark.\+cpp.

\hypertarget{class_c_benchmark_a43dd8d9d01499e28f1dc899ea6f3ed97}{}\index{C\+Benchmark@{C\+Benchmark}!start\+Timer@{start\+Timer}}
\index{start\+Timer@{start\+Timer}!C\+Benchmark@{C\+Benchmark}}
\paragraph[{start\+Timer}]{\setlength{\rightskip}{0pt plus 5cm}L\+A\+R\+G\+E\+\_\+\+I\+N\+T\+E\+G\+E\+R C\+Benchmark\+::start\+Timer (
\begin{DoxyParamCaption}
{}
\end{DoxyParamCaption}
)\hspace{0.3cm}{\ttfamily [private]}}\label{class_c_benchmark_a43dd8d9d01499e28f1dc899ea6f3ed97}


Definicja w linii 10 pliku benchmark.\+cpp.

\hypertarget{class_c_benchmark_a945aaa453776cd11395166b470d11778}{}\index{C\+Benchmark@{C\+Benchmark}!stop\+\_\+timer@{stop\+\_\+timer}}
\index{stop\+\_\+timer@{stop\+\_\+timer}!C\+Benchmark@{C\+Benchmark}}
\paragraph[{stop\+\_\+timer}]{\setlength{\rightskip}{0pt plus 5cm}void C\+Benchmark\+::stop\+\_\+timer (
\begin{DoxyParamCaption}
{}
\end{DoxyParamCaption}
)\hspace{0.3cm}{\ttfamily [virtual]}}\label{class_c_benchmark_a945aaa453776cd11395166b470d11778}


Definicja w linii 33 pliku benchmark.\+cpp.



\subsubsection{Dokumentacja atrybutów składowych}
\hypertarget{class_c_benchmark_ae72cb7dad8bf32730109776d546af2ad}{}\index{C\+Benchmark@{C\+Benchmark}!performance\+Count\+End@{performance\+Count\+End}}
\index{performance\+Count\+End@{performance\+Count\+End}!C\+Benchmark@{C\+Benchmark}}
\paragraph[{performance\+Count\+End}]{\setlength{\rightskip}{0pt plus 5cm}L\+A\+R\+G\+E\+\_\+\+I\+N\+T\+E\+G\+E\+R C\+Benchmark\+::performance\+Count\+End\hspace{0.3cm}{\ttfamily [private]}}\label{class_c_benchmark_ae72cb7dad8bf32730109776d546af2ad}


Definicja w linii 18 pliku benchmark.\+hh.

\hypertarget{class_c_benchmark_a54ec1bb4f1e570341e14adfc6312f1b7}{}\index{C\+Benchmark@{C\+Benchmark}!performance\+Count\+Start@{performance\+Count\+Start}}
\index{performance\+Count\+Start@{performance\+Count\+Start}!C\+Benchmark@{C\+Benchmark}}
\paragraph[{performance\+Count\+Start}]{\setlength{\rightskip}{0pt plus 5cm}L\+A\+R\+G\+E\+\_\+\+I\+N\+T\+E\+G\+E\+R C\+Benchmark\+::performance\+Count\+Start\hspace{0.3cm}{\ttfamily [private]}}\label{class_c_benchmark_a54ec1bb4f1e570341e14adfc6312f1b7}


Definicja w linii 17 pliku benchmark.\+hh.



Dokumentacja dla tej klasy została wygenerowana z plików\+:\begin{DoxyCompactItemize}
\item 
\hyperlink{benchmark_8hh}{benchmark.\+hh}\item 
\hyperlink{benchmark_8cpp}{benchmark.\+cpp}\end{DoxyCompactItemize}

\hypertarget{class_c_heap_sort}{}\subsection{C\+Heap\+Sort Class Reference}
\label{class_c_heap_sort}\index{C\+Heap\+Sort@{C\+Heap\+Sort}}


definicja klasy \hyperlink{class_c_heap_sort}{C\+Heap\+Sort} definiuje sortowanie przez kopcowanie przyklad klasy obserwowanej  




{\ttfamily \#include $<$heap\+\_\+sort.\+hh$>$}



Inherits \hyperlink{class_c_sort}{C\+Sort}.

\subsubsection*{Public Member Functions}
\begin{DoxyCompactItemize}
\item 
void \hyperlink{class_c_heap_sort_af859772c7f50bb824b66cc6ec11b72b0}{sorting} (\hyperlink{class_c_list}{C\+List} $\ast$list, int useless, int last)
\begin{DoxyCompactList}\small\item\em definicja metody sorting sortuje wykorzystujac algorytm heapsort tworzy kopiec zamienia najwiekszy element z najwiekszym(wysyla najwiekszy na koniec listy przywraca wlasciwosc kopca korzystajac z metody up\+\_\+heap \end{DoxyCompactList}\item 
void \hyperlink{class_c_heap_sort_a31374a3fb22fdab6dfdd0ca0db1c5973}{sort} (\hyperlink{class_c_list}{C\+List} $\ast$list, int useless, int last)
\begin{DoxyCompactList}\small\item\em definicja metody sort implemetacja metody abstakcyjnej wykorzystuje heapsort z wykorzystaniem timerow \end{DoxyCompactList}\item 
void \hyperlink{class_c_heap_sort_a388bdc396f69e678ea3f3e5a86a09aae}{build\+\_\+heap} (\hyperlink{class_c_list}{C\+List} $\ast$list, int last)
\begin{DoxyCompactList}\small\item\em definicja metody build\+\_\+heap tworzy kopiec z listy wykorzystuje metode down\+\_\+heap \end{DoxyCompactList}\item 
void \hyperlink{class_c_heap_sort_afb9acabdd2839321148916259f347a24}{down\+\_\+heap} (\hyperlink{class_c_list}{C\+List} $\ast$list, int parent, int last)
\begin{DoxyCompactList}\small\item\em definicja metody max\+\_\+heapify przywraca wlasciwosc kopca zakladajac korzen -\/ najwiekszy element \end{DoxyCompactList}\item 
void \hyperlink{class_c_heap_sort_a4b2aa9fa2e1f6284c72a943ce3f84dac}{up\+\_\+heap} (\hyperlink{class_c_list}{C\+List} $\ast$list, int last)
\begin{DoxyCompactList}\small\item\em definicja metody up\+\_\+heap przywraca wlasciwosc kopca zakladajac korzen -\/ najwiekszy element \end{DoxyCompactList}\item 
void \hyperlink{class_c_heap_sort_ae5ef55ad47223dc1308ce9ffbe11648a}{benchmarking} (\hyperlink{class_c_list}{C\+List} $\ast$list)
\begin{DoxyCompactList}\small\item\em feinicja metody benchmarking sortuje listy w zakresie 1-\/ 10 000 wykonujac metode sort, zapisuje dane z licznikw czasu do pliku \end{DoxyCompactList}\end{DoxyCompactItemize}


\subsubsection{Detailed Description}
definicja klasy \hyperlink{class_c_heap_sort}{C\+Heap\+Sort} definiuje sortowanie przez kopcowanie przyklad klasy obserwowanej 

\subsubsection{Member Function Documentation}
\hypertarget{class_c_heap_sort_ae5ef55ad47223dc1308ce9ffbe11648a}{}\index{C\+Heap\+Sort@{C\+Heap\+Sort}!benchmarking@{benchmarking}}
\index{benchmarking@{benchmarking}!C\+Heap\+Sort@{C\+Heap\+Sort}}
\paragraph[{benchmarking}]{\setlength{\rightskip}{0pt plus 5cm}void C\+Heap\+Sort\+::benchmarking (
\begin{DoxyParamCaption}
\item[{{\bf C\+List} $\ast$}]{list}
\end{DoxyParamCaption}
)\hspace{0.3cm}{\ttfamily [virtual]}}\label{class_c_heap_sort_ae5ef55ad47223dc1308ce9ffbe11648a}


feinicja metody benchmarking sortuje listy w zakresie 1-\/ 10 000 wykonujac metode sort, zapisuje dane z licznikw czasu do pliku 


\begin{DoxyParams}{Parameters}
{\em list} & -\/ benchmarkowana lista \\
\hline
\end{DoxyParams}


Implements \hyperlink{class_c_sort_a600cf20261b3e00b148f8d4b773dd1b9}{C\+Sort}.

\hypertarget{class_c_heap_sort_a388bdc396f69e678ea3f3e5a86a09aae}{}\index{C\+Heap\+Sort@{C\+Heap\+Sort}!build\+\_\+heap@{build\+\_\+heap}}
\index{build\+\_\+heap@{build\+\_\+heap}!C\+Heap\+Sort@{C\+Heap\+Sort}}
\paragraph[{build\+\_\+heap}]{\setlength{\rightskip}{0pt plus 5cm}void C\+Heap\+Sort\+::build\+\_\+heap (
\begin{DoxyParamCaption}
\item[{{\bf C\+List} $\ast$}]{list, }
\item[{int}]{last}
\end{DoxyParamCaption}
)}\label{class_c_heap_sort_a388bdc396f69e678ea3f3e5a86a09aae}


definicja metody build\+\_\+heap tworzy kopiec z listy wykorzystuje metode down\+\_\+heap 


\begin{DoxyParams}{Parameters}
{\em list} & -\/ modyfkowana lista \\
\hline
{\em last} & -\/ maksymalny rozmiar kopca \\
\hline
\end{DoxyParams}
\hypertarget{class_c_heap_sort_afb9acabdd2839321148916259f347a24}{}\index{C\+Heap\+Sort@{C\+Heap\+Sort}!down\+\_\+heap@{down\+\_\+heap}}
\index{down\+\_\+heap@{down\+\_\+heap}!C\+Heap\+Sort@{C\+Heap\+Sort}}
\paragraph[{down\+\_\+heap}]{\setlength{\rightskip}{0pt plus 5cm}void C\+Heap\+Sort\+::down\+\_\+heap (
\begin{DoxyParamCaption}
\item[{{\bf C\+List} $\ast$}]{list, }
\item[{int}]{parent, }
\item[{int}]{last}
\end{DoxyParamCaption}
)}\label{class_c_heap_sort_afb9acabdd2839321148916259f347a24}


definicja metody max\+\_\+heapify przywraca wlasciwosc kopca zakladajac korzen -\/ najwiekszy element 


\begin{DoxyParams}{Parameters}
{\em list} & -\/ modyfikowana lista \\
\hline
{\em parent} & -\/ \textquotesingle{}korzen poddrzewa\textquotesingle{} \\
\hline
{\em last} & -\/ maksymalny rozmiar drzewa algorytm przyraca wlasciwosc kopca zaczynajac od korzenia \\
\hline
\end{DoxyParams}
\hypertarget{class_c_heap_sort_a31374a3fb22fdab6dfdd0ca0db1c5973}{}\index{C\+Heap\+Sort@{C\+Heap\+Sort}!sort@{sort}}
\index{sort@{sort}!C\+Heap\+Sort@{C\+Heap\+Sort}}
\paragraph[{sort}]{\setlength{\rightskip}{0pt plus 5cm}void C\+Heap\+Sort\+::sort (
\begin{DoxyParamCaption}
\item[{{\bf C\+List} $\ast$}]{list, }
\item[{int}]{useless, }
\item[{int}]{last}
\end{DoxyParamCaption}
)\hspace{0.3cm}{\ttfamily [virtual]}}\label{class_c_heap_sort_a31374a3fb22fdab6dfdd0ca0db1c5973}


definicja metody sort implemetacja metody abstakcyjnej wykorzystuje heapsort z wykorzystaniem timerow 


\begin{DoxyParams}{Parameters}
{\em list} & -\/ sortowana lista \\
\hline
{\em useless} & -\/ useless \\
\hline
{\em last} & -\/ maksymany romziar listy \\
\hline
\end{DoxyParams}


Implements \hyperlink{class_c_sort_a2c87a533501c9e2102444e4ce6b69527}{C\+Sort}.

\hypertarget{class_c_heap_sort_af859772c7f50bb824b66cc6ec11b72b0}{}\index{C\+Heap\+Sort@{C\+Heap\+Sort}!sorting@{sorting}}
\index{sorting@{sorting}!C\+Heap\+Sort@{C\+Heap\+Sort}}
\paragraph[{sorting}]{\setlength{\rightskip}{0pt plus 5cm}void C\+Heap\+Sort\+::sorting (
\begin{DoxyParamCaption}
\item[{{\bf C\+List} $\ast$}]{list, }
\item[{int}]{useless, }
\item[{int}]{last}
\end{DoxyParamCaption}
)}\label{class_c_heap_sort_af859772c7f50bb824b66cc6ec11b72b0}


definicja metody sorting sortuje wykorzystujac algorytm heapsort tworzy kopiec zamienia najwiekszy element z najwiekszym(wysyla najwiekszy na koniec listy przywraca wlasciwosc kopca korzystajac z metody up\+\_\+heap 


\begin{DoxyParams}{Parameters}
{\em list} & -\/ sortowana lista \\
\hline
{\em useless} & -\/ useless \\
\hline
{\em last} & -\/ maksymany romziar listy \\
\hline
\end{DoxyParams}
\hypertarget{class_c_heap_sort_a4b2aa9fa2e1f6284c72a943ce3f84dac}{}\index{C\+Heap\+Sort@{C\+Heap\+Sort}!up\+\_\+heap@{up\+\_\+heap}}
\index{up\+\_\+heap@{up\+\_\+heap}!C\+Heap\+Sort@{C\+Heap\+Sort}}
\paragraph[{up\+\_\+heap}]{\setlength{\rightskip}{0pt plus 5cm}void C\+Heap\+Sort\+::up\+\_\+heap (
\begin{DoxyParamCaption}
\item[{{\bf C\+List} $\ast$}]{list, }
\item[{int}]{last}
\end{DoxyParamCaption}
)}\label{class_c_heap_sort_a4b2aa9fa2e1f6284c72a943ce3f84dac}


definicja metody up\+\_\+heap przywraca wlasciwosc kopca zakladajac korzen -\/ najwiekszy element 


\begin{DoxyParams}{Parameters}
{\em list} & -\/ modyfikowana lista \\
\hline
{\em last} & -\/ maksymalny rozmiar kopca \\
\hline
\end{DoxyParams}


The documentation for this class was generated from the following files\+:\begin{DoxyCompactItemize}
\item 
\hyperlink{heap__sort_8hh}{heap\+\_\+sort.\+hh}\item 
\hyperlink{heap__sort_8cpp}{heap\+\_\+sort.\+cpp}\end{DoxyCompactItemize}

\hypertarget{class_c_list}{}\subsection{C\+List Class Reference}
\label{class_c_list}\index{C\+List@{C\+List}}


klasa lista -\/ A\+D\+T modeluje prost� liste jednokierunkowa zwiera metody niezbedne do implementacji sortowania  




{\ttfamily \#include $<$list.\+hh$>$}

\subsubsection*{Public Member Functions}
\begin{DoxyCompactItemize}
\item 
\hyperlink{class_c_list_ae6434f79bc2d54fbd3617214a313af97}{C\+List} ()
\item 
\hyperlink{class_c_list_a34a40e8ab96400a4ec186b29de56da5a}{$\sim$\+C\+List} ()
\item 
void \hyperlink{class_c_list_abcf376342791b5a361dbb16b2f3c18e7}{print} () const 
\begin{DoxyCompactList}\small\item\em definicja funkcji wyswietlajacej wyswietla na strumieniu wyjsciowym ciag elementow zapisanych na liscie \end{DoxyCompactList}\item 
void \hyperlink{class_c_list_a5cac185c7d3bddaef565557c3b2fbc52}{push} (int element)
\begin{DoxyCompactList}\small\item\em definicja metody push dodaje nowy element na liste \end{DoxyCompactList}\item 
int \hyperlink{class_c_list_ac62395d2e755a624dca890266679c0e4}{pop} ()
\begin{DoxyCompactList}\small\item\em definicja metody pop usuwa element z lsity \end{DoxyCompactList}\item 
int \hyperlink{class_c_list_a47ef53b4cfb22f2bd03021f73259c7f3}{get\+\_\+value} (int i) const 
\begin{DoxyCompactList}\small\item\em definicja metody get\+\_\+value odowluje sie do wybranego elemntu listy \end{DoxyCompactList}\item 
int \& \hyperlink{class_c_list_a95c54088a50836e6ae48b041c98b1019}{get\+\_\+value} (int i)
\begin{DoxyCompactList}\small\item\em definicja przeciazenia get\+\_\+value \end{DoxyCompactList}\item 
void \hyperlink{class_c_list_a8da32b78baee180c05201e2efa67622e}{swap} (int i, int j)
\begin{DoxyCompactList}\small\item\em definicja metody swap zamienia elementy listy \end{DoxyCompactList}\item 
bool \hyperlink{class_c_list_a317015b03da2b120c97bb27f8cd744c0}{is\+\_\+empty} ()
\begin{DoxyCompactList}\small\item\em definicja metody is\+\_\+empty \end{DoxyCompactList}\item 
int \hyperlink{class_c_list_af2a1d2a860b6d8f2a93077c8571ad3a1}{get\+\_\+size} ()
\begin{DoxyCompactList}\small\item\em definicja metody get\+\_\+size \end{DoxyCompactList}\item 
void \hyperlink{class_c_list_a3c96b4974b942fe5b7867a4fc2e4d277}{pull} (int i)
\begin{DoxyCompactList}\small\item\em definicja metody pull wypelnia liste liczbami pesudolosowymi \end{DoxyCompactList}\end{DoxyCompactItemize}
\subsubsection*{Private Attributes}
\begin{DoxyCompactItemize}
\item 
int \hyperlink{class_c_list_a72784f3a1dd907527c1f4398582faa26}{value}
\item 
\hyperlink{class_c_list}{C\+List} $\ast$ \hyperlink{class_c_list_a623dc92ae6bd3d600508527c0db3ddbf}{next}
\end{DoxyCompactItemize}


\subsubsection{Detailed Description}
klasa lista -\/ A\+D\+T modeluje prost� liste jednokierunkowa zwiera metody niezbedne do implementacji sortowania 

\subsubsection{Constructor \& Destructor Documentation}
\hypertarget{class_c_list_ae6434f79bc2d54fbd3617214a313af97}{}\index{C\+List@{C\+List}!C\+List@{C\+List}}
\index{C\+List@{C\+List}!C\+List@{C\+List}}
\paragraph[{C\+List}]{\setlength{\rightskip}{0pt plus 5cm}C\+List\+::\+C\+List (
\begin{DoxyParamCaption}
{}
\end{DoxyParamCaption}
)}\label{class_c_list_ae6434f79bc2d54fbd3617214a313af97}
\hypertarget{class_c_list_a34a40e8ab96400a4ec186b29de56da5a}{}\index{C\+List@{C\+List}!````~C\+List@{$\sim$\+C\+List}}
\index{````~C\+List@{$\sim$\+C\+List}!C\+List@{C\+List}}
\paragraph[{$\sim$\+C\+List}]{\setlength{\rightskip}{0pt plus 5cm}C\+List\+::$\sim$\+C\+List (
\begin{DoxyParamCaption}
{}
\end{DoxyParamCaption}
)}\label{class_c_list_a34a40e8ab96400a4ec186b29de56da5a}


\subsubsection{Member Function Documentation}
\hypertarget{class_c_list_af2a1d2a860b6d8f2a93077c8571ad3a1}{}\index{C\+List@{C\+List}!get\+\_\+size@{get\+\_\+size}}
\index{get\+\_\+size@{get\+\_\+size}!C\+List@{C\+List}}
\paragraph[{get\+\_\+size}]{\setlength{\rightskip}{0pt plus 5cm}int C\+List\+::get\+\_\+size (
\begin{DoxyParamCaption}
{}
\end{DoxyParamCaption}
)}\label{class_c_list_af2a1d2a860b6d8f2a93077c8571ad3a1}


definicja metody get\+\_\+size 

\begin{DoxyReturn}{Returns}
ilosc elementow na liscie 
\end{DoxyReturn}
\hypertarget{class_c_list_a47ef53b4cfb22f2bd03021f73259c7f3}{}\index{C\+List@{C\+List}!get\+\_\+value@{get\+\_\+value}}
\index{get\+\_\+value@{get\+\_\+value}!C\+List@{C\+List}}
\paragraph[{get\+\_\+value}]{\setlength{\rightskip}{0pt plus 5cm}int C\+List\+::get\+\_\+value (
\begin{DoxyParamCaption}
\item[{int}]{i}
\end{DoxyParamCaption}
) const}\label{class_c_list_a47ef53b4cfb22f2bd03021f73259c7f3}


definicja metody get\+\_\+value odowluje sie do wybranego elemntu listy 


\begin{DoxyParams}{Parameters}
{\em i} & -\/ indeks kom�rki listy \\
\hline
\end{DoxyParams}
\begin{DoxyReturn}{Returns}
element o indeksie i 
\end{DoxyReturn}
\hypertarget{class_c_list_a95c54088a50836e6ae48b041c98b1019}{}\index{C\+List@{C\+List}!get\+\_\+value@{get\+\_\+value}}
\index{get\+\_\+value@{get\+\_\+value}!C\+List@{C\+List}}
\paragraph[{get\+\_\+value}]{\setlength{\rightskip}{0pt plus 5cm}int \& C\+List\+::get\+\_\+value (
\begin{DoxyParamCaption}
\item[{int}]{i}
\end{DoxyParamCaption}
)}\label{class_c_list_a95c54088a50836e6ae48b041c98b1019}


definicja przeciazenia get\+\_\+value 

\begin{DoxyReturn}{Returns}
referencje do elementu na lsicie 
\end{DoxyReturn}
\hypertarget{class_c_list_a317015b03da2b120c97bb27f8cd744c0}{}\index{C\+List@{C\+List}!is\+\_\+empty@{is\+\_\+empty}}
\index{is\+\_\+empty@{is\+\_\+empty}!C\+List@{C\+List}}
\paragraph[{is\+\_\+empty}]{\setlength{\rightskip}{0pt plus 5cm}bool C\+List\+::is\+\_\+empty (
\begin{DoxyParamCaption}
{}
\end{DoxyParamCaption}
)}\label{class_c_list_a317015b03da2b120c97bb27f8cd744c0}


definicja metody is\+\_\+empty 

\begin{DoxyReturn}{Returns}
true -\/ gdy lista jest pusta 

false -\/ w przypadku przeciwnym 
\end{DoxyReturn}
\hypertarget{class_c_list_ac62395d2e755a624dca890266679c0e4}{}\index{C\+List@{C\+List}!pop@{pop}}
\index{pop@{pop}!C\+List@{C\+List}}
\paragraph[{pop}]{\setlength{\rightskip}{0pt plus 5cm}int C\+List\+::pop (
\begin{DoxyParamCaption}
{}
\end{DoxyParamCaption}
)}\label{class_c_list_ac62395d2e755a624dca890266679c0e4}


definicja metody pop usuwa element z lsity 

\begin{DoxyReturn}{Returns}
usuwany element 
\end{DoxyReturn}
\hypertarget{class_c_list_abcf376342791b5a361dbb16b2f3c18e7}{}\index{C\+List@{C\+List}!print@{print}}
\index{print@{print}!C\+List@{C\+List}}
\paragraph[{print}]{\setlength{\rightskip}{0pt plus 5cm}void C\+List\+::print (
\begin{DoxyParamCaption}
{}
\end{DoxyParamCaption}
) const}\label{class_c_list_abcf376342791b5a361dbb16b2f3c18e7}


definicja funkcji wyswietlajacej wyswietla na strumieniu wyjsciowym ciag elementow zapisanych na liscie 

\hypertarget{class_c_list_a3c96b4974b942fe5b7867a4fc2e4d277}{}\index{C\+List@{C\+List}!pull@{pull}}
\index{pull@{pull}!C\+List@{C\+List}}
\paragraph[{pull}]{\setlength{\rightskip}{0pt plus 5cm}void C\+List\+::pull (
\begin{DoxyParamCaption}
\item[{int}]{i}
\end{DoxyParamCaption}
)}\label{class_c_list_a3c96b4974b942fe5b7867a4fc2e4d277}


definicja metody pull wypelnia liste liczbami pesudolosowymi 


\begin{DoxyParams}{Parameters}
{\em i} & -\/ ilosc elementow \\
\hline
\end{DoxyParams}
\hypertarget{class_c_list_a5cac185c7d3bddaef565557c3b2fbc52}{}\index{C\+List@{C\+List}!push@{push}}
\index{push@{push}!C\+List@{C\+List}}
\paragraph[{push}]{\setlength{\rightskip}{0pt plus 5cm}void C\+List\+::push (
\begin{DoxyParamCaption}
\item[{int}]{element}
\end{DoxyParamCaption}
)}\label{class_c_list_a5cac185c7d3bddaef565557c3b2fbc52}


definicja metody push dodaje nowy element na liste 


\begin{DoxyParams}{Parameters}
{\em element} & -\/ dodawana komorka do listy \\
\hline
\end{DoxyParams}
\hypertarget{class_c_list_a8da32b78baee180c05201e2efa67622e}{}\index{C\+List@{C\+List}!swap@{swap}}
\index{swap@{swap}!C\+List@{C\+List}}
\paragraph[{swap}]{\setlength{\rightskip}{0pt plus 5cm}void C\+List\+::swap (
\begin{DoxyParamCaption}
\item[{int}]{i, }
\item[{int}]{j}
\end{DoxyParamCaption}
)}\label{class_c_list_a8da32b78baee180c05201e2efa67622e}


definicja metody swap zamienia elementy listy 


\begin{DoxyParams}{Parameters}
{\em i} & -\/ inkeks do pierwszego elementu \\
\hline
{\em j} & -\/ indeks do drugiego elementu \\
\hline
\end{DoxyParams}


\subsubsection{Member Data Documentation}
\hypertarget{class_c_list_a623dc92ae6bd3d600508527c0db3ddbf}{}\index{C\+List@{C\+List}!next@{next}}
\index{next@{next}!C\+List@{C\+List}}
\paragraph[{next}]{\setlength{\rightskip}{0pt plus 5cm}{\bf C\+List}$\ast$ C\+List\+::next\hspace{0.3cm}{\ttfamily [private]}}\label{class_c_list_a623dc92ae6bd3d600508527c0db3ddbf}
\hypertarget{class_c_list_a72784f3a1dd907527c1f4398582faa26}{}\index{C\+List@{C\+List}!value@{value}}
\index{value@{value}!C\+List@{C\+List}}
\paragraph[{value}]{\setlength{\rightskip}{0pt plus 5cm}int C\+List\+::value\hspace{0.3cm}{\ttfamily [private]}}\label{class_c_list_a72784f3a1dd907527c1f4398582faa26}


The documentation for this class was generated from the following files\+:\begin{DoxyCompactItemize}
\item 
\hyperlink{list_8hh}{list.\+hh}\item 
\hyperlink{list_8cpp}{list.\+cpp}\end{DoxyCompactItemize}

\hypertarget{class_c_merge_sort}{}\subsection{C\+Merge\+Sort Class Reference}
\label{class_c_merge_sort}\index{C\+Merge\+Sort@{C\+Merge\+Sort}}


definicja klasy \hyperlink{class_c_merge_sort}{C\+Merge\+Sort} definiuje sortowanie przez scalanie jest przykadem klasy obserwowanej implementuje metode abstakcyjna sort  




{\ttfamily \#include $<$merge\+\_\+sort.\+hh$>$}



Inherits \hyperlink{class_c_sort}{C\+Sort}.

\subsubsection*{Public Member Functions}
\begin{DoxyCompactItemize}
\item 
void \hyperlink{class_c_merge_sort_a93fee6618c382c29957c35affabd9e6b}{sorting} (\hyperlink{class_c_list}{C\+List} $\ast$list, int left, int right)
\begin{DoxyCompactList}\small\item\em definicja metody sorting sortuje liste poprzez algorytm sortowania przez scalanie \end{DoxyCompactList}\item 
void \hyperlink{class_c_merge_sort_a5becc6aef876cfc13abe746dea5a4f70}{sort} (\hyperlink{class_c_list}{C\+List} $\ast$list, int left, int right)
\begin{DoxyCompactList}\small\item\em definicja metody sort implementacja metody czysto abstrakcyjnej korzysta z metody sorting korzysta z timerow \end{DoxyCompactList}\item 
void \hyperlink{class_c_merge_sort_add3bb6f2c5f822eb1d435a0652bdf431}{benchmarking} (\hyperlink{class_c_list}{C\+List} $\ast$list)
\begin{DoxyCompactList}\small\item\em feinicja metody benchmarking sortuje listy w zakresie 1-\/ 10 000 wykonujac metode sort, zapisuje dane z licznikw czasu do pliku \end{DoxyCompactList}\end{DoxyCompactItemize}


\subsubsection{Detailed Description}
definicja klasy \hyperlink{class_c_merge_sort}{C\+Merge\+Sort} definiuje sortowanie przez scalanie jest przykadem klasy obserwowanej implementuje metode abstakcyjna sort 

\subsubsection{Member Function Documentation}
\hypertarget{class_c_merge_sort_add3bb6f2c5f822eb1d435a0652bdf431}{}\index{C\+Merge\+Sort@{C\+Merge\+Sort}!benchmarking@{benchmarking}}
\index{benchmarking@{benchmarking}!C\+Merge\+Sort@{C\+Merge\+Sort}}
\paragraph[{benchmarking}]{\setlength{\rightskip}{0pt plus 5cm}void C\+Merge\+Sort\+::benchmarking (
\begin{DoxyParamCaption}
\item[{{\bf C\+List} $\ast$}]{list}
\end{DoxyParamCaption}
)\hspace{0.3cm}{\ttfamily [virtual]}}\label{class_c_merge_sort_add3bb6f2c5f822eb1d435a0652bdf431}


feinicja metody benchmarking sortuje listy w zakresie 1-\/ 10 000 wykonujac metode sort, zapisuje dane z licznikw czasu do pliku 


\begin{DoxyParams}{Parameters}
{\em list} & -\/ benchmarkowana lista \\
\hline
\end{DoxyParams}


Implements \hyperlink{class_c_sort_a600cf20261b3e00b148f8d4b773dd1b9}{C\+Sort}.

\hypertarget{class_c_merge_sort_a5becc6aef876cfc13abe746dea5a4f70}{}\index{C\+Merge\+Sort@{C\+Merge\+Sort}!sort@{sort}}
\index{sort@{sort}!C\+Merge\+Sort@{C\+Merge\+Sort}}
\paragraph[{sort}]{\setlength{\rightskip}{0pt plus 5cm}void C\+Merge\+Sort\+::sort (
\begin{DoxyParamCaption}
\item[{{\bf C\+List} $\ast$}]{list, }
\item[{int}]{left, }
\item[{int}]{right}
\end{DoxyParamCaption}
)\hspace{0.3cm}{\ttfamily [virtual]}}\label{class_c_merge_sort_a5becc6aef876cfc13abe746dea5a4f70}


definicja metody sort implementacja metody czysto abstrakcyjnej korzysta z metody sorting korzysta z timerow 


\begin{DoxyParams}{Parameters}
{\em list} & -\/ sortowana lista \\
\hline
{\em left} & -\/ indeks na 1-\/szy element \\
\hline
{\em right} & -\/ indeks na ostatni element sortowanej listy \\
\hline
\end{DoxyParams}


Implements \hyperlink{class_c_sort_a2c87a533501c9e2102444e4ce6b69527}{C\+Sort}.

\hypertarget{class_c_merge_sort_a93fee6618c382c29957c35affabd9e6b}{}\index{C\+Merge\+Sort@{C\+Merge\+Sort}!sorting@{sorting}}
\index{sorting@{sorting}!C\+Merge\+Sort@{C\+Merge\+Sort}}
\paragraph[{sorting}]{\setlength{\rightskip}{0pt plus 5cm}void C\+Merge\+Sort\+::sorting (
\begin{DoxyParamCaption}
\item[{{\bf C\+List} $\ast$}]{list, }
\item[{int}]{left, }
\item[{int}]{right}
\end{DoxyParamCaption}
)}\label{class_c_merge_sort_a93fee6618c382c29957c35affabd9e6b}


definicja metody sorting sortuje liste poprzez algorytm sortowania przez scalanie 


\begin{DoxyParams}{Parameters}
{\em list} & -\/ sortowana lista \\
\hline
{\em left} & -\/ indeks na 1-\/szy element \\
\hline
{\em right} & -\/ indeks na ostatni element sortowanej listy \\
\hline
\end{DoxyParams}


The documentation for this class was generated from the following files\+:\begin{DoxyCompactItemize}
\item 
\hyperlink{merge__sort_8hh}{merge\+\_\+sort.\+hh}\item 
\hyperlink{merge__sort_8cpp}{merge\+\_\+sort.\+cpp}\end{DoxyCompactItemize}

\hypertarget{class_c_quick_sort}{}\subsection{C\+Quick\+Sort Class Reference}
\label{class_c_quick_sort}\index{C\+Quick\+Sort@{C\+Quick\+Sort}}


defnijca klasy \hyperlink{class_c_sort}{C\+Sort} definiuje sortowanie szybkie jest przykaladem klasy obserwowanej  




{\ttfamily \#include $<$quick\+\_\+sort.\+hh$>$}



Inherits \hyperlink{class_c_sort}{C\+Sort}.

\subsubsection*{Public Member Functions}
\begin{DoxyCompactItemize}
\item 
void \hyperlink{class_c_quick_sort_a9918875ff4f1ca8430d761e5f3ac4a9c}{sorting} (\hyperlink{class_c_list}{C\+List} $\ast$list, int left, int right)
\begin{DoxyCompactList}\small\item\em defnicja metody sorting implementacja algortymu quicksort bez timerow \end{DoxyCompactList}\item 
void \hyperlink{class_c_quick_sort_a94cdaa98ea0a6ce6b1c214466f311d97}{sort} (\hyperlink{class_c_list}{C\+List} $\ast$list, int left, int right)
\begin{DoxyCompactList}\small\item\em definicja metody sort implementacja algorytmu quicksort z wykorzystaniem timerow \end{DoxyCompactList}\item 
void \hyperlink{class_c_quick_sort_ac60e09cc11c7c08dfc3ffc75d43502a7}{benchmarking} (\hyperlink{class_c_list}{C\+List} $\ast$list)
\begin{DoxyCompactList}\small\item\em feinicja metody benchmarking sortuje listy w zakresie 1-\/ 10 000 wykonujac metode sort, zapisuje dane z licznikw czasu do pliku \end{DoxyCompactList}\end{DoxyCompactItemize}


\subsubsection{Detailed Description}
defnijca klasy \hyperlink{class_c_sort}{C\+Sort} definiuje sortowanie szybkie jest przykaladem klasy obserwowanej 

\subsubsection{Member Function Documentation}
\hypertarget{class_c_quick_sort_ac60e09cc11c7c08dfc3ffc75d43502a7}{}\index{C\+Quick\+Sort@{C\+Quick\+Sort}!benchmarking@{benchmarking}}
\index{benchmarking@{benchmarking}!C\+Quick\+Sort@{C\+Quick\+Sort}}
\paragraph[{benchmarking}]{\setlength{\rightskip}{0pt plus 5cm}void C\+Quick\+Sort\+::benchmarking (
\begin{DoxyParamCaption}
\item[{{\bf C\+List} $\ast$}]{list}
\end{DoxyParamCaption}
)\hspace{0.3cm}{\ttfamily [virtual]}}\label{class_c_quick_sort_ac60e09cc11c7c08dfc3ffc75d43502a7}


feinicja metody benchmarking sortuje listy w zakresie 1-\/ 10 000 wykonujac metode sort, zapisuje dane z licznikw czasu do pliku 


\begin{DoxyParams}{Parameters}
{\em list} & -\/ benchmarkowana lista \\
\hline
\end{DoxyParams}


Implements \hyperlink{class_c_sort_a600cf20261b3e00b148f8d4b773dd1b9}{C\+Sort}.

\hypertarget{class_c_quick_sort_a94cdaa98ea0a6ce6b1c214466f311d97}{}\index{C\+Quick\+Sort@{C\+Quick\+Sort}!sort@{sort}}
\index{sort@{sort}!C\+Quick\+Sort@{C\+Quick\+Sort}}
\paragraph[{sort}]{\setlength{\rightskip}{0pt plus 5cm}void C\+Quick\+Sort\+::sort (
\begin{DoxyParamCaption}
\item[{{\bf C\+List} $\ast$}]{list, }
\item[{int}]{left, }
\item[{int}]{right}
\end{DoxyParamCaption}
)\hspace{0.3cm}{\ttfamily [virtual]}}\label{class_c_quick_sort_a94cdaa98ea0a6ce6b1c214466f311d97}


definicja metody sort implementacja algorytmu quicksort z wykorzystaniem timerow 


\begin{DoxyParams}{Parameters}
{\em list} & -\/ sortowana lista \\
\hline
{\em left} & -\/ indeks na 1 element sortowanej listy \\
\hline
{\em right} & -\/ indeks na prawy element sortowanej listy \\
\hline
\end{DoxyParams}


Implements \hyperlink{class_c_sort_a2c87a533501c9e2102444e4ce6b69527}{C\+Sort}.

\hypertarget{class_c_quick_sort_a9918875ff4f1ca8430d761e5f3ac4a9c}{}\index{C\+Quick\+Sort@{C\+Quick\+Sort}!sorting@{sorting}}
\index{sorting@{sorting}!C\+Quick\+Sort@{C\+Quick\+Sort}}
\paragraph[{sorting}]{\setlength{\rightskip}{0pt plus 5cm}void C\+Quick\+Sort\+::sorting (
\begin{DoxyParamCaption}
\item[{{\bf C\+List} $\ast$}]{list, }
\item[{int}]{left, }
\item[{int}]{right}
\end{DoxyParamCaption}
)}\label{class_c_quick_sort_a9918875ff4f1ca8430d761e5f3ac4a9c}


defnicja metody sorting implementacja algortymu quicksort bez timerow 


\begin{DoxyParams}{Parameters}
{\em list} & -\/ sortowana lista \\
\hline
{\em left} & -\/ indeks na 1 element sortowanej listy \\
\hline
{\em right} & -\/ indeks na prawy element sortowanej listy \\
\hline
\end{DoxyParams}


The documentation for this class was generated from the following files\+:\begin{DoxyCompactItemize}
\item 
\hyperlink{quick__sort_8hh}{quick\+\_\+sort.\+hh}\item 
\hyperlink{quick__sort_8cpp}{quick\+\_\+sort.\+cpp}\end{DoxyCompactItemize}

\hypertarget{class_c_sort}{}\subsection{C\+Sort Class Reference}
\label{class_c_sort}\index{C\+Sort@{C\+Sort}}


definicja klasy abstrakcyjnej \hyperlink{class_c_sort}{C\+Sort} jest klasa bazowa dla konkretnych typow sortowan. jest przykladem klasy obserwowanej.  




{\ttfamily \#include $<$csort.\+hh$>$}



Inherits \hyperlink{class_c_benchmark}{C\+Benchmark}.



Inherited by \hyperlink{class_c_heap_sort}{C\+Heap\+Sort}, \hyperlink{class_c_merge_sort}{C\+Merge\+Sort}, and \hyperlink{class_c_quick_sort}{C\+Quick\+Sort}.

\subsubsection*{Public Member Functions}
\begin{DoxyCompactItemize}
\item 
virtual void \hyperlink{class_c_sort_a2c87a533501c9e2102444e4ce6b69527}{sort} (\hyperlink{class_c_list}{C\+List} $\ast$, int, int)=0
\begin{DoxyCompactList}\small\item\em definicja metody sort sortuje elementy na liscie. \end{DoxyCompactList}\item 
virtual void \hyperlink{class_c_sort_a600cf20261b3e00b148f8d4b773dd1b9}{benchmarking} (\hyperlink{class_c_list}{C\+List} $\ast$list)=0
\end{DoxyCompactItemize}


\subsubsection{Detailed Description}
definicja klasy abstrakcyjnej \hyperlink{class_c_sort}{C\+Sort} jest klasa bazowa dla konkretnych typow sortowan. jest przykladem klasy obserwowanej. 

\subsubsection{Member Function Documentation}
\hypertarget{class_c_sort_a600cf20261b3e00b148f8d4b773dd1b9}{}\index{C\+Sort@{C\+Sort}!benchmarking@{benchmarking}}
\index{benchmarking@{benchmarking}!C\+Sort@{C\+Sort}}
\paragraph[{benchmarking}]{\setlength{\rightskip}{0pt plus 5cm}virtual void C\+Sort\+::benchmarking (
\begin{DoxyParamCaption}
\item[{{\bf C\+List} $\ast$}]{list}
\end{DoxyParamCaption}
)\hspace{0.3cm}{\ttfamily [pure virtual]}}\label{class_c_sort_a600cf20261b3e00b148f8d4b773dd1b9}


Implemented in \hyperlink{class_c_heap_sort_ae5ef55ad47223dc1308ce9ffbe11648a}{C\+Heap\+Sort}, \hyperlink{class_c_merge_sort_add3bb6f2c5f822eb1d435a0652bdf431}{C\+Merge\+Sort}, and \hyperlink{class_c_quick_sort_ac60e09cc11c7c08dfc3ffc75d43502a7}{C\+Quick\+Sort}.

\hypertarget{class_c_sort_a2c87a533501c9e2102444e4ce6b69527}{}\index{C\+Sort@{C\+Sort}!sort@{sort}}
\index{sort@{sort}!C\+Sort@{C\+Sort}}
\paragraph[{sort}]{\setlength{\rightskip}{0pt plus 5cm}virtual void C\+Sort\+::sort (
\begin{DoxyParamCaption}
\item[{{\bf C\+List} $\ast$}]{, }
\item[{int}]{, }
\item[{int}]{}
\end{DoxyParamCaption}
)\hspace{0.3cm}{\ttfamily [pure virtual]}}\label{class_c_sort_a2c87a533501c9e2102444e4ce6b69527}


definicja metody sort sortuje elementy na liscie. 


\begin{DoxyParams}{Parameters}
{\em list} & -\/ sortowana lista \\
\hline
{\em left} & -\/ indeks pierwszego elementu listy sortowanej. \\
\hline
{\em right} & -\/ indeks ostatniego elementu listy sortowanej. \\
\hline
\end{DoxyParams}


Implemented in \hyperlink{class_c_heap_sort_a31374a3fb22fdab6dfdd0ca0db1c5973}{C\+Heap\+Sort}, \hyperlink{class_c_merge_sort_a5becc6aef876cfc13abe746dea5a4f70}{C\+Merge\+Sort}, and \hyperlink{class_c_quick_sort_a94cdaa98ea0a6ce6b1c214466f311d97}{C\+Quick\+Sort}.



The documentation for this class was generated from the following file\+:\begin{DoxyCompactItemize}
\item 
\hyperlink{csort_8hh}{csort.\+hh}\end{DoxyCompactItemize}

\section{File Documentation}
\hypertarget{benchmark_8cpp}{}\subsection{Dokumentacja pliku benchmark.\+cpp}
\label{benchmark_8cpp}\index{benchmark.\+cpp@{benchmark.\+cpp}}
{\ttfamily \#include \char`\"{}benchmark.\+hh\char`\"{}}\\*
{\ttfamily \#include $<$windows.\+h$>$}\\*
{\ttfamily \#include $<$fstream$>$}\\*
{\ttfamily \#include $<$iostream$>$}\\*

\hypertarget{benchmark_8hh}{}\subsection{Dokumentacja pliku benchmark.\+hh}
\label{benchmark_8hh}\index{benchmark.\+hh@{benchmark.\+hh}}
{\ttfamily \#include $<$windows.\+h$>$}\\*
\subsubsection*{Komponenty}
\begin{DoxyCompactItemize}
\item 
class \hyperlink{class_c_benchmark}{C\+Benchmark}
\end{DoxyCompactItemize}

\hypertarget{csort_8hh}{}\subsection{csort.\+hh File Reference}
\label{csort_8hh}\index{csort.\+hh@{csort.\+hh}}
{\ttfamily \#include \char`\"{}list.\+hh\char`\"{}}\\*
{\ttfamily \#include \char`\"{}benchmark.\+hh\char`\"{}}\\*
\subsubsection*{Classes}
\begin{DoxyCompactItemize}
\item 
class \hyperlink{class_c_sort}{C\+Sort}
\begin{DoxyCompactList}\small\item\em definicja klasy abstrakcyjnej \hyperlink{class_c_sort}{C\+Sort} jest klasa bazowa dla konkretnych typow sortowan. jest przykladem klasy obserwowanej. \end{DoxyCompactList}\end{DoxyCompactItemize}

\hypertarget{heap__sort_8cpp}{}\subsection{heap\+\_\+sort.\+cpp File Reference}
\label{heap__sort_8cpp}\index{heap\+\_\+sort.\+cpp@{heap\+\_\+sort.\+cpp}}
{\ttfamily \#include \char`\"{}heap\+\_\+sort.\+hh\char`\"{}}\\*

\hypertarget{heap__sort_8hh}{}\subsection{heap\+\_\+sort.\+hh File Reference}
\label{heap__sort_8hh}\index{heap\+\_\+sort.\+hh@{heap\+\_\+sort.\+hh}}
{\ttfamily \#include \char`\"{}list.\+hh\char`\"{}}\\*
{\ttfamily \#include \char`\"{}csort.\+hh\char`\"{}}\\*
\subsubsection*{Classes}
\begin{DoxyCompactItemize}
\item 
class \hyperlink{class_c_heap_sort}{C\+Heap\+Sort}
\begin{DoxyCompactList}\small\item\em definicja klasy \hyperlink{class_c_heap_sort}{C\+Heap\+Sort} definiuje sortowanie przez kopcowanie przyklad klasy obserwowanej \end{DoxyCompactList}\end{DoxyCompactItemize}

\hypertarget{list_8cpp}{}\subsection{list.\+cpp File Reference}
\label{list_8cpp}\index{list.\+cpp@{list.\+cpp}}
{\ttfamily \#include \char`\"{}list.\+hh\char`\"{}}\\*
{\ttfamily \#include $<$iostream$>$}\\*
{\ttfamily \#include $<$cstdlib$>$}\\*

\hypertarget{list_8hh}{}\subsection{list.\+hh File Reference}
\label{list_8hh}\index{list.\+hh@{list.\+hh}}
\subsubsection*{Classes}
\begin{DoxyCompactItemize}
\item 
class \hyperlink{class_c_list}{C\+List}
\begin{DoxyCompactList}\small\item\em klasa lista -\/ A\+D\+T modeluje prost� liste jednokierunkowa zwiera metody niezbedne do implementacji sortowania \end{DoxyCompactList}\end{DoxyCompactItemize}

\hypertarget{main_8cpp}{}\subsection{Dokumentacja pliku main.\+cpp}
\label{main_8cpp}\index{main.\+cpp@{main.\+cpp}}
{\ttfamily \#include $<$iostream$>$}\\*
{\ttfamily \#include $<$cstdlib$>$}\\*
{\ttfamily \#include \char`\"{}losowy\+\_\+lancuch.\+hh\char`\"{}}\\*
{\ttfamily \#include \char`\"{}haszowanie.\+hh\char`\"{}}\\*
{\ttfamily \#include \char`\"{}dane.\+hh\char`\"{}}\\*
{\ttfamily \#include \char`\"{}benchmark.\+hh\char`\"{}}\\*
{\ttfamily \#include $<$ctime$>$}\\*
{\ttfamily \#include $<$fstream$>$}\\*
\subsubsection*{Funkcje}
\begin{DoxyCompactItemize}
\item 
int \hyperlink{main_8cpp_ae66f6b31b5ad750f1fe042a706a4e3d4}{main} ()
\end{DoxyCompactItemize}


\subsubsection{Dokumentacja funkcji}
\hypertarget{main_8cpp_ae66f6b31b5ad750f1fe042a706a4e3d4}{}\index{main.\+cpp@{main.\+cpp}!main@{main}}
\index{main@{main}!main.\+cpp@{main.\+cpp}}
\paragraph[{main}]{\setlength{\rightskip}{0pt plus 5cm}int main (
\begin{DoxyParamCaption}
{}
\end{DoxyParamCaption}
)}\label{main_8cpp_ae66f6b31b5ad750f1fe042a706a4e3d4}


Definicja w linii 26 pliku main.\+cpp.


\hypertarget{merge__sort_8cpp}{}\subsection{merge\+\_\+sort.\+cpp File Reference}
\label{merge__sort_8cpp}\index{merge\+\_\+sort.\+cpp@{merge\+\_\+sort.\+cpp}}
{\ttfamily \#include $<$iostream$>$}\\*
{\ttfamily \#include \char`\"{}list.\+hh\char`\"{}}\\*
{\ttfamily \#include \char`\"{}merge\+\_\+sort.\+hh\char`\"{}}\\*

\hypertarget{merge__sort_8hh}{}\subsection{merge\+\_\+sort.\+hh File Reference}
\label{merge__sort_8hh}\index{merge\+\_\+sort.\+hh@{merge\+\_\+sort.\+hh}}
{\ttfamily \#include \char`\"{}list.\+hh\char`\"{}}\\*
{\ttfamily \#include \char`\"{}csort.\+hh\char`\"{}}\\*
\subsubsection*{Classes}
\begin{DoxyCompactItemize}
\item 
class \hyperlink{class_c_merge_sort}{C\+Merge\+Sort}
\begin{DoxyCompactList}\small\item\em definicja klasy \hyperlink{class_c_merge_sort}{C\+Merge\+Sort} definiuje sortowanie przez scalanie jest przykadem klasy obserwowanej implementuje metode abstakcyjna sort \end{DoxyCompactList}\end{DoxyCompactItemize}

\hypertarget{quick__sort_8cpp}{}\subsection{quick\+\_\+sort.\+cpp File Reference}
\label{quick__sort_8cpp}\index{quick\+\_\+sort.\+cpp@{quick\+\_\+sort.\+cpp}}
{\ttfamily \#include \char`\"{}quick\+\_\+sort.\+hh\char`\"{}}\\*

\hypertarget{quick__sort_8hh}{}\subsection{quick\+\_\+sort.\+hh File Reference}
\label{quick__sort_8hh}\index{quick\+\_\+sort.\+hh@{quick\+\_\+sort.\+hh}}
{\ttfamily \#include \char`\"{}list.\+hh\char`\"{}}\\*
{\ttfamily \#include \char`\"{}benchmark.\+hh\char`\"{}}\\*
{\ttfamily \#include \char`\"{}csort.\+hh\char`\"{}}\\*
\subsubsection*{Classes}
\begin{DoxyCompactItemize}
\item 
class \hyperlink{class_c_quick_sort}{C\+Quick\+Sort}
\begin{DoxyCompactList}\small\item\em defnijca klasy \hyperlink{class_c_sort}{C\+Sort} definiuje sortowanie szybkie jest przykaladem klasy obserwowanej \end{DoxyCompactList}\end{DoxyCompactItemize}

%\documentclass[11pt,a4paper]{article}
%\usepackage{polski}
%\usepackage[utf8]{inputenc}

\title{Projektowanie algorytmów i metod sztucznej inteligencji\\Laboratorium 7 - Sprawozdanie}
\author{Wojciech Makuch}
\date{}

%\begin{document}
	\maketitle
	\section{Zadanie}\label{sec:Zadanie}
	Jako zadanie na zajęcia należało zmodyfikować implementację licznika czasu tak,  aby korzystały one ze wzorca projektowania: obserwator. Ponadto należało zmodyfikować algorytmy sortowania, aby zachowane zostały zasady SOLID, a w szczególności zasada  \textsl{Open/Closed principle} - program otwarty na rozszerzanie, ale zamknięty na modyfikację.
	
	\section{Realizacja}\label{sec:Realizacja}
	Większość klas, metod i funkcji zmodyfikowano zmieniając ich nazwy, nazwy zmiennych itp. aby były napisane w Języku angielskim. Strukturę danych lista napisano od początku, ponieważ stara zawierała błąd w implementacji przy próbie odniesienia się do pierwszego elementu, poza tym brakowało jej niezbędnych metod, takich jak np. \textsl{get\_value()}.  Jeśli chodzi o sortowania, zgodnie z zaleceniem prowadzącego utworzono klasę abstrakcyjną po której dziedziczyły klasy \textsl{CQuickSort}, \textsl{CHeapSort} oraz \textsl{CMergeSort}, każda z nich implementowała metodę sort.  Zmodyfikowano implementację licznika czasu; teraz korzysta ze wzorca projektowego obserwator za pomocą mechanizmu dziedziczenia. Obserwuje klasę abstrakcyjną \textsl{CSort}, w której wykorzystano metody benchmarkujące \textsl{star\_timer()} i \textsl{stop\_timer()}.  Po wykonaniu sortowania  metoda klasy Benchmark zapisuję do pliku ilość elementów sortowanych oraz czas sortowania. Ponadto dodano do funkcji głównej programu menu dla użytkownika, które pozwala na sprawdzenie poprawności zaimplementowanych metod.   
	
	\section{Dzialanie}\label{sec:Dzialanie}
	Wszystkie algorytmy działają prawidłowo. Dane do pliku dopisywane są po każdym sortowaniu. Plik trzeba usuwać ręcznie. Dzięki metodzie benchmarking nie trzeba recznie testowac sortowania kazdej listy.
	
	\section{Komentarz}\label{sec:Komentarz}
	Do utworzenia dokumentacji wykorzystano system Doxygen. Funkcja pomiaru czasu dla systemu Windows pobrana ze strony dr. J. Mierzwy. Program skompilowano w środowisku Code::Blocks. Do stworzenia wykresu posłużono się pakietem MS Excel, sprawozdanie napisano używając systemu \LaTeX.

\end{document}